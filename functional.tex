%  -*- coding: utf-8 -*-

\documentclass[oneside]{book}
\usepackage[a4paper,margin=2.5cm]{geometry}

\setlength{\parindent}{0pt}
\setlength{\parskip}{4pt plus 1pt minus 1pt}

\usepackage{functional}

\Functional{scoping=false,tracing=true}

\begin{document}

\chapter{Overview of Features}

\ExplSyntaxOn

\PrgNewFunction \MathSquare { m }
  {
    \IntSet \lTmpaInt { \IntEval { #1 * #1 } }
    \Result { \IntUse \lTmpaInt }
  }

\PrgNewFunction \MathCubic { m }
  {
    \IntSet \lTmpaInt { \IntEval { #1 * #1 * #1 } }
    \Result { \Value \lTmpaInt }
  }

\PrgNewFunction \Factorial { m }
  {
    \IntCompareTF { #1 } = { 0 }
      { \Result { 1 } }
      {
        \TlSet \lTmpaTl
          {
            \IntMathMult { #1 } { \Factorial { \IntMathSub{#1}{1} } }
          }
        \Result { \Value \lTmpaTl }
      }
  }

\ExplSyntaxOff

\MathSquare{5}
\MathSquare{\MathSquare{5}}

\MathCubic{2}
\MathCubic{\MathCubic{2}}

\Factorial{0}
\Factorial{4}

\chapter{Basic Definitions (\texttt{l3basics})}

\GobbleOne{abc} \UseOne{uvw}
\UseGobble{abc}{uvw} \GobbleUse{abc}{uvw}

\chapter{Control Structures (\texttt{l3prg})}

\BoolIfTF\cTrueBool{\Result{true}}{\Result{false}}
\BoolIfTF\cFalseBool{\Result{true}}{\Result{false}}

\BoolSetTrue\lTmpaBool
\BoolIfTF\lTmpaBool{\Result{true}}{\Result{false}}
\BoolSetFalse\lTmpaBool
\BoolIfTF\lTmpaBool{\Result{true}}{\Result{false}}

\chapter{Token Lists (\texttt{l3tl})}

\TlSet{\lTmpiTl}{\IntMathMult{4}{5}}
\TlUse{\lTmpiTl}

\TlSet{\lTmpjTl}{ijk}
\TlClear{\lTmpjTl}
\TlUse{\lTmpjTl}
\TlSet{\lTmpjTl}{uvw}
\TlUse{\lTmpjTl}

\TlSet{\lTmpkTl}{Functional}
\TlPutLeft{\lTmpkTl}{Hello}
\TlPutRight{\lTmpkTl}{World}
\TlUse{\lTmpkTl}

\TlSet\lTmpaTl{abc}
\TlIfEmptyTF\lTmpaTl{\Result{empty}}{\Result{nonempty}}
\TlClear\lTmpaTl
\TlIfEmptyTF\lTmpaTl{\Result{empty}}{\Result{nonempty}}

\TlSet{\lTmpaTl}{abc}
\TlSet{\lTmpbTl}{abc}
\TlSet{\lTmpcTl}{xyz}
\TlIfEqTF{\lTmpaTl}{\lTmpbTl}{\Result{first}}{\Result{second}}
\TlIfEqTF{\lTmpaTl}{\lTmpcTl}{\Result{first}}{\Result{second}}

\chapter{Integers (\texttt{l3int})}

\IntSet\lTmpaInt{5}
\IntUse\lTmpaInt

\IntSet\lTmpaInt{5}
\IntZero\lTmpaInt
\IntUse\lTmpaInt

\IntSet\lTmpaInt{5}
\IntIncr\lTmpaInt
\IntUse\lTmpaInt

\IntSet\lTmpaInt{5}
\IntDecr\lTmpaInt
\IntUse\lTmpaInt

\IntSet\lTmpaInt{5}
\IntAdd\lTmpaInt{2}
\IntUse\lTmpaInt

\IntSet\lTmpaInt{5}
\IntSub\lTmpaInt{3}
\IntUse\lTmpaInt

\IntMathAdd{7}{3}
\IntMathSub{7}{3}
\IntMathMult{7}{3}
\IntMathDiv{7}{3}

\IntCompareTF{2}>{1}{\Result{first}}{\Result{second}}
\IntCompareTF{2}>{3}{\Result{first}}{\Result{second}}

\end{document}

