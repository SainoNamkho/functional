% -*- coding: utf-8 -*-
% !TEX program = lualatex
\documentclass[oneside]{book}

% -*- coding: utf-8 -*-
% !TEX program = lualatex

\usepackage[a4paper,margin=2.5cm]{geometry}

\newcommand*{\myversion}{2022F}
\newcommand*{\mydate}{Version \myversion\ (\the\year-\mylpad\month-\mylpad\day)}
\newcommand*{\mylpad}[1]{\ifnum#1<10 0\the#1\else\the#1\fi}

\setlength{\parindent}{0pt}
\setlength{\parskip}{4pt plus 1pt minus 1pt}

\usepackage{codehigh}

\colorlet{highback}{blue9}
%\CodeHigh{lite}
\CodeHigh{language=latex/latex2,style/main=highback,style/code=highback}
\NewCodeHighEnv{code}{style/main=gray9,style/code=gray9}
\NewCodeHighEnv{demo}{style/main=gray9,style/code=gray9,demo}

\usepackage{enumitem}

\NewDocumentCommand\MySubScript{m}{$_{#1}$}

\ExplSyntaxOn
\NewDocumentCommand\PrintVarList{m}{
  \clist_set:Nn \l_tmpa_clist {#1}
  \clist_map_inline:Nn \l_tmpa_clist
    {
      \token_to_str:N ##1 ~
    }
}
\NewDocumentCommand\RelaceChacters{m}{
  \tl_set:Nn \lTmpaTl {#1}
  \regex_replace_once:nnN { \_ } { \c{MySubScript} } \lTmpaTl
}
\NewDocumentCommand\RelaceUnderScoreAll{m}{
  \tl_set:Nn \lTmpaTl {#1}
  \regex_replace_all:nnN { \_ } { \c{_} } \lTmpaTl
  \lTmpaTl
}
\ExplSyntaxOff

\NewDocumentEnvironment{variable}{om}{
  \vspace{5pt}
  \begin{minipage}{\linewidth}
  \hrule\vspace{4pt}\obeylines%
  \begingroup
  \ttfamily\bfseries\color{azure3}
  \PrintVarList{#2}
  \endgroup
  \par\vspace{4pt}\hrule
  \end{minipage}\par\nopagebreak\vspace{4pt}
}{%
  \vspace{5pt}%
}

\NewDocumentEnvironment{function}{om}{
  \vspace{5pt}%
}{\vspace{5pt}}

\NewDocumentEnvironment{syntax}{}{%
  \begin{minipage}{\linewidth}
  \hrule\vspace{4pt}\obeylines%
}{%
  \par\vspace{4pt}\hrule
  \end{minipage}\par\nopagebreak\vspace{4pt}
}

\NewDocumentEnvironment{texnote}{}{}{}

\NewDocumentCommand\cs{O{}m}{%
  \texttt{\bfseries\color{purple3}\cBackslashStr\RelaceUnderScoreAll{#2}}%
}
\NewDocumentCommand\meta{m}{%
  \RelaceChacters{#1}%
  \textsl{$\langle$\ignorespaces\lTmpaTl\unskip$\rangle$}%
}
\NewDocumentCommand\Arg{m}{%
  \RelaceChacters{#1}%
  \texttt{\{}\textsl{$\langle$\ignorespaces\lTmpaTl\unskip$\rangle$}\texttt{\}}%
}

\NewDocumentCommand\pkg{m}{\textsf{#1}}

\NewDocumentCommand\nan{}{\texttt{NaN}}
\NewDocumentCommand\enquote{m}{``#1''}

\let\tn=\cs

\RenewDocumentCommand\emph{m}{%
  \underline{\textsl{#1}}%
}

\newenvironment{l3regex-syntax}
  {\begin{itemize}\def\\{\char`\\}\def\makelabel##1{\hss\llap{\ttfamily##1}}}
  {\end{itemize}}

\usepackage{shortvrb}

\NewDocumentCommand \MyMakeShortVerb {} {%
  \MakeShortVerb \|%
  \MakeShortVerb \"%
}
\NewDocumentCommand \MyDeleteShortVerb {} {%
  \DeleteShortVerb \|%
  \DeleteShortVerb \"%
}

\AtBeginDocument{\MyMakeShortVerb}
\AtEndDocument{\MyDeleteShortVerb}
\AddToHook{env/demohigh/before}{\MyDeleteShortVerb}
\AddToHook{env/demohigh/after}{\MyMakeShortVerb}

\usepackage{hologo}
\providecommand*\LuaTeX{\hologo{LuaTeX}}
\providecommand*\pdfTeX{\hologo{pdfTeX}}
\providecommand*\XeTeX{\hologo{XeTeX}}

\usepackage{amsmath}

\usepackage{hyperref}
\hypersetup{
  colorlinks=true,
  urlcolor=blue3,
  linkcolor=blue3,
}

\usepackage{functional}
%\Functional{scoping=false,tracing=true}

\CodeHigh{lite}

\begin{document}

\chapter{Integers (\texttt{Int})}

\section{Constant and Scratch Integers}

\begin{variable}{\cZeroInt,\cOneInt}
Integer values used with primitive tests and assignments: their
self-terminating nature makes these more convenient and faster than
literal numbers.
\end{variable}

\begin{variable}{\cMaxInt}
The maximum value that can be stored as an integer.
\end{variable}

\begin{variable}{\cMaxRegisterInt}
Maximum number of registers.
\end{variable}

\begin{variable}{\cMaxCharInt}
Maximum character code completely supported by the engine.
\end{variable}

\begin{variable}{\lTmpaInt,\lTmpbInt,\lTmpcInt,\lTmpiInt,\lTmpjInt,\lTmpkInt}
Scratch integer for local assignment. These are never used by
the \verb!functional! package, and so are safe for use with any
function. However, they may be overwritten by other
code and so should only be used for short-term storage.
\end{variable}

\begin{variable}{\gTmpaInt,\gTmpbInt,\gTmpcInt,\gTmpiInt,\gTmpjInt,\gTmpkInt}
Scratch integer for global assignment. These are never used by
the \verb!functional! package, and so are safe for use with any
function. However, they may be overwritten by other
code and so should only be used for short-term storage.
\end{variable}

\section{Integer Expressions}

\begin{function}{\IntEval}
\begin{syntax}
\cs{IntEval} \Arg{integer expression}
\end{syntax}
Evaluates the \meta{integer expression} and returns the result:
for positive results an
explicit sequence of decimal digits not starting with~\texttt{0},
for negative results \texttt{-}~followed by such a sequence, and
\texttt{0}~for zero. For example
\begin{demohigh}
\IntEval {(1+4)*(2-3)/5}
\end{demohigh}
\end{function}

\begin{function}{\IntMathAdd}
\begin{syntax}
\cs{IntMathAdd} \Arg{integer expression_1} \Arg{integer expression_2}
\end{syntax}
Adds \Arg{integer expression_1} and \Arg{integer expression_2},
and returns the result. For example
\begin{demohigh}
\IntMathAdd {7} {3}
\end{demohigh}
\end{function}

\begin{function}{\IntMathSub}
\begin{syntax}
\cs{IntMathSub} \Arg{integer expression_1} \Arg{integer expression_2}
\end{syntax}
Subtracts \Arg{integer expression_2} from \Arg{integer expression_1},
and returns the result. For example
\begin{demohigh}
\IntMathSub {7} {3}
\end{demohigh}
\end{function}

\begin{function}{\IntMathMult}
\begin{syntax}
\cs{IntMathMult} \Arg{integer expression_1} \Arg{integer expression_2}
\end{syntax}
Multiplies \Arg{integer expression_1} by \Arg{integer expression_2},
and returns the result. For example
\begin{demohigh}
\IntMathMult {7} {3}
\end{demohigh}
\end{function}

\begin{function}{\IntMathDiv}
\begin{syntax}
\cs{IntMathDiv} \Arg{integer expression_1} \Arg{integer expression_2}
\end{syntax}
Evaluates the two \meta{integer expressions} as described earlier,
then divides the first value by the second, and rounds the result
to the closest integer. Ties are rounded away from zero. Note that
this is identical to using \verb|/| directly in an \meta{integer expression}.
The result is returned as an \meta{integer denotation}.
For example
\begin{demohigh}
\IntMathDiv {8} {3}
\end{demohigh}
\end{function}

\begin{function}{\IntMathDivTrancate}
\begin{syntax}
\cs{IntMathDivTrancate} \Arg{integer expression_1} \Arg{integer expression_2}
\end{syntax}
Evaluates the two \meta{integer expressions} as described earlier,
then divides the first value by the second, and rounds the result towards zero.
Note that division using \verb|/| rounds to the closest integer instead.
The result is returned as an \meta{integer denotation}.
For example
\begin{demohigh}
\IntMathDivTruncate {8} {3}
\end{demohigh}
\end{function}

\begin{function}{\IntMathSign}
\begin{syntax}
\cs{IntMathSign} \Arg{intexpr}
\end{syntax}
Evaluates the \meta{integer expression} then leaves $1$ or $0$ or
$-1$ in the input stream according to the sign of the result.
\end{function}

\begin{function}{\IntMathAbs}
\begin{syntax}
\cs{IntMathAbs} \Arg{integer expression}
\end{syntax}
Evaluates the \meta{integer expression} as described for
\cs{IntEval} and leaves the absolute value of the result in
the input stream as an \meta{integer denotation} after two
expansions.
\end{function}

\begin{function}{\IntMathMax,\IntMathMin}
\begin{syntax}
\cs{IntMathMax} \Arg{intexpr_1} \Arg{intexpr_2}
\cs{IntMathMin} \Arg{intexpr_1} \Arg{intexpr_2}
\end{syntax}
Evaluates the \meta{integer expressions} as described for
\cs{IntEval} and leaves either the larger or smaller value
in the input stream as an \meta{integer denotation} after two
expansions.
\end{function}

\begin{function}{\IntMathMod}
\begin{syntax}
\cs{IntMathMod} \Arg{intexpr_1} \Arg{intexpr_2}
\end{syntax}
Evaluates the two \meta{integer expressions} as described earlier,
then calculates the integer remainder of dividing the first
expression by the second.  This is obtained by subtracting
\cs{IntMathDivTruncate} \Arg{intexpr_1} \Arg{intexpr_2} times
\meta{intexpr_2} from \meta{intexpr_1}.  Thus, the result has the
same sign as \meta{intexpr_1} and its absolute value is strictly
less than that of \meta{intexpr_2}.  The result is left in the input
stream as an \meta{integer denotation} after two expansions.
\end{function}

\begin{function}{\IntMathRand}
\begin{syntax}
\cs{IntMathRand} \Arg{intexpr_1} \Arg{intexpr_2}
\end{syntax}
Evaluates the two \meta{integer expressions} and produces a
pseudo-random number between the two (with bounds included).
%This is not available in older versions of \XeTeX{}.
\end{function}

\section{Creating and Using Integers}

\begin{function}{\IntNew}
\begin{syntax}
\cs{IntNew} \meta{integer}
\end{syntax}
Creates a new \meta{integer} or raises an error if the name is
already taken. The declaration is global. The \meta{integer} is
initially equal to $0$.
\end{function}

\begin{function}{\IntConst}
\begin{syntax}
\cs{IntConst} \meta{integer} \Arg{integer expression}
\end{syntax}
Creates a new constant \meta{integer} or raises an error if the name
is already taken. The value of the \meta{integer} is set
globally to the \meta{integer expression}.
\end{function}

\begin{function}{\IntUse}
\begin{syntax}
\cs{IntUse} \meta{integer}
\end{syntax}
Recovers the content of an \meta{integer} and returns the value.
An error is raised if the variable does not exist or if it is invalid.
\end{function}

\section{Viewing Integers}

\begin{function}{\IntLog}
\begin{syntax}
\cs{IntLog} \Arg{integer expression}
\end{syntax}
Writes the result of evaluating the \meta{integer expression}
in the log file.
\end{function}

\begin{function}{\IntVarLog}
\begin{syntax}
\cs{IntVarLog} \meta{integer}
\end{syntax}
Writes the value of the \meta{integer} in the log file.
\end{function}

\begin{function}{\IntShow}
\begin{syntax}
\cs{IntShow} \Arg{integer expression}
\end{syntax}
Displays the result of evaluating the \meta{integer expression}
on the terminal.
\end{function}

\begin{function}{\IntVarShow}
\begin{syntax}
\cs{IntVarShow} \meta{integer}
\end{syntax}
Displays the value of the \meta{integer} on the terminal.
\end{function}

\section{Setting Integer Variables}

\begin{function}{\IntSet}
\begin{syntax}
\cs{IntSet} \meta{integer} \Arg{integer expression}
\end{syntax}
Sets \meta{integer} to the value of \meta{integer expression},
which must evaluate to an integer (as described for \cs{IntEval}).
For example
\begin{demohigh}
\IntSet \lTmpaInt {3+5}
\IntUse \lTmpaInt
\end{demohigh}
\end{function}

\begin{function}{\IntZero}
\begin{syntax}
\cs{IntZero} \meta{integer}
\end{syntax}
Sets \meta{integer} to $0$. For example
\begin{demohigh}
\IntSet \lTmpaInt {5}
\IntZero \lTmpaInt
\IntUse \lTmpaInt
\end{demohigh}
\end{function}

\begin{function}{\IntZeroNew}
\begin{syntax}
\cs{IntZeroNew} \meta{integer}
\end{syntax}
Ensures that the \meta{integer} exists globally by applying
\cs{IntNew} if necessary, then applies \cs{IntZero} to leave
the \meta{integer} set to zero.
\end{function}

\begin{function}{\IntSetEq}
\begin{syntax}
\cs{IntSetEq} \meta{integer_1} \meta{integer_2}
\end{syntax}
Sets the content of \meta{integer_1} equal to that of
\meta{integer_2}.
\end{function}

\begin{function}{\IntIncr}
\begin{syntax}
\cs{IntIncr} \meta{integer}
\end{syntax}
Increases the value stored in \meta{integer} by $1$.
For example
\begin{demohigh}
\IntSet \lTmpaInt {5}
\IntIncr \lTmpaInt
\IntUse \lTmpaInt
\end{demohigh}
\end{function}

\begin{function}{\IntDecr}
\begin{syntax}
\cs{IntDecr} \meta{integer}
\end{syntax}
Decreases the value stored in \meta{integer} by $1$.
For example
\begin{demohigh}
\IntSet \lTmpaInt {5}
\IntDecr \lTmpaInt
\IntUse \lTmpaInt
\end{demohigh}
\end{function}

\begin{function}{\IntAdd}
\begin{syntax}
\cs{IntAdd} \meta{integer} \Arg{integer expression}
\end{syntax}
Adds the result of the \meta{integer expression} to the current
content of the \meta{integer}. For example
\begin{demohigh}
\IntSet \lTmpaInt {5}
\IntAdd \lTmpaInt {2}
\IntUse \lTmpaInt
\end{demohigh}
\end{function}

\begin{function}{\IntSub}
\begin{syntax}
\cs{IntSub} \meta{integer} \Arg{integer expression}
\end{syntax}
Subtracts the result of the \meta{integer expression} from the
current content of the \meta{integer}. For example
\begin{demohigh}
\IntSet \lTmpaInt {5}
\IntSub \lTmpaInt {3}
\IntUse \lTmpaInt
\end{demohigh}
\end{function}

\section{Integer Step Functions}

%\begin{function}{\IntStepFunction}
%\begin{syntax}
%\cs{IntStepFunction} \Arg{initial value} \Arg{step} \Arg{final value} \meta{function}
%\end{syntax}
%This function first evaluates the \meta{initial value}, \meta{step}
%and \meta{final value}, all of which should be integer expressions.
%The \meta{function} is then placed in front of each \meta{value}
%from the \meta{initial value} to the \meta{final value} in turn
%(using \meta{step} between each \meta{value}).  The \meta{step} must
%be non-zero.  If the \meta{step} is positive, the loop stops when
%the \meta{value} becomes larger than the \meta{final value}.  If the
%\meta{step} is negative, the loop stops when the \meta{value}
%becomes smaller than the \meta{final value}.  The \meta{function}
%should absorb one numerical argument. For example
%\begin{verbatim}
%\cs_set:Npn \my_func:n #1 { [I~saw~#1] \quad }
%\int_step_function:nnnN { 1 } { 1 } { 5 } \my_func:n
%\end{verbatim}
%would print
%\begin{quote}
%[I saw 1] \quad
%[I saw 2] \quad
%[I saw 3] \quad
%[I saw 4] \quad
%[I saw 5] \quad
%\end{quote}
%\end{function}

\begin{function}{\IntStepInline}
\begin{syntax}
\cs{IntStepInline} \Arg{initial value} \Arg{step} \Arg{final value} \Arg{code}
\end{syntax}
This function first evaluates the \meta{initial value}, \meta{step}
and \meta{final value}, all of which should be integer expressions.
Then for each \meta{value} from the \meta{initial value} to the
\meta{final value} in turn (using \meta{step} between each
\meta{value}), the \meta{code} is inserted into the input stream
with \verb|#1| replaced by the current \meta{value}. Thus the
\meta{code} should define a function of one argument~(\verb|#1|).
For example
\begin{codehigh}
\IgnoreSpacesOn
\TlClear \lTmpaTl
\IntStepInline {1} {3} {30} {
  \TlPutRight \lTmpaTl {[#1]}
}
\Result {\Value\lTmpaTl}
\IgnoreSpacesOff
\end{codehigh}
produces [1][4][7][10][13][16][19][22][25][28].
\end{function}

\begin{function}{\IntStepVariable}
\begin{syntax}
\cs{IntStepVariable} \Arg{initial value} \Arg{step} \Arg{final value} \meta{tl var} \Arg{code}
\end{syntax}
This function first evaluates the \meta{initial value}, \meta{step}
and \meta{final value}, all of which should be integer expressions.
Then for each \meta{value} from the \meta{initial value} to the
\meta{final value} in turn (using \meta{step} between each
\meta{value}), the \meta{code} is evaluated,
with the \meta{tl~var} defined as the current \meta{value}. Thus
the \meta{code} should make use of the \meta{tl~var}.
%For example
%\begin{demohigh}
%\IgnoreSpacesOn
%\TlClear\lTmpaTl
%\IntStepVariable{1}{3}{30}\lTmpiTl{
%  \TlPutRight\lTmpaTl{\Value\lTmpiTl}
%  \TlPutRight\lTmpaTl{~}
%}
%\Result{\Value\lTmpaTl}
%\IgnoreSpacesOff
%\end{demohigh}
\end{function}

\section{Integer Conditionals}

\begin{function}{\IntIfExist, \IntIfExistTF}
\begin{syntax}
\cs{IntIfExist} \meta{integer}
\cs{IntIfExistTF} \meta{integer} \Arg{true code} \Arg{false code}
\end{syntax}
Tests whether the \meta{integer} is currently defined.  This does not
check that the \meta{integer} really is an integer variable.
\end{function}

\begin{function}{\IntIfOdd,\IntIfOddTF}
\begin{syntax}
\cs{IntIfOdd} \Arg{integer expression}
\cs{IntIfOddTF} \Arg{integer expression} \Arg{true code} \Arg{false code}
\end{syntax}
This function first evaluates the \meta{integer expression}
as described for \cs{IntEval}.
It then evaluates if this is odd or even, as appropriate.
\end{function}

\begin{function}{\IntIfEven,\IntIfEvenTF}
\begin{syntax}
\cs{IntIfEven} \Arg{integer expression}
\cs{IntIfEvenTF} \Arg{integer expression} \Arg{true code} \Arg{false code}
\end{syntax}
This function first evaluates the \meta{integer expression}
as described for \cs{IntEval}.
It then evaluates if this is even or odd, as appropriate.
\end{function}

\begin{function}{\IntCompare,\IntCompareTF}
\begin{syntax}
\cs{IntCompare} \Arg{intexpr_1} \meta{relation} \Arg{intexpr_2}
\cs{IntCompareTF} \Arg{intexpr_1} \meta{relation} \Arg{intexpr_2} \Arg{true code} \Arg{false code}
\end{syntax}
This function first evaluates each of the \meta{integer expressions}
as described for \cs{IntEval}. The two results are then
compared using the \meta{relation}:\par
{\centering\begin{tabular}{ll}
Equal        & \texttt{=} \\
Greater than & \texttt{>} \\
Less than    & \texttt{<} \\
\end{tabular}\par}
For example
\begin{demohigh}
\IntCompareTF {2} > {1} {\Result{Greater}} {\Result{Less}}
\IntCompareTF {2} > {3} {\Result{Greater}} {\Result{Less}}
\end{demohigh}
\end{function}

\section{Integer Case Functions}

\begin{function}{\IntCase}
\begin{syntax}
\cs{IntCase} \Arg{test integer expression}
~ ~ \verb|{|
~ ~ ~ ~ \Arg{intexpr case_1} \Arg{code case_1}
~ ~ ~ ~ \Arg{intexpr case_2} \Arg{code case_2}
~ ~ ~ ~ $\cdots$
~ ~ ~ ~ \Arg{intexpr case_n} \Arg{code case_n}
~ ~ \verb|}|
\end{syntax}
This function evaluates the \meta{test integer expression} and
compares this in turn to each of the
\meta{integer expression cases}. If the two are equal then the
associated \meta{code} is left in the input stream
and other cases are discarded.
\end{function}

\begin{function}{\IntCaseT}
\begin{syntax}
\cs{IntCaseT} \Arg{test integer expression}
~ ~ \verb|{|
~ ~ ~ ~ \Arg{intexpr case_1} \Arg{code case_1}
~ ~ ~ ~ \Arg{intexpr case_2} \Arg{code case_2}
~ ~ ~ ~ $\cdots$
~ ~ ~ ~ \Arg{intexpr case_n} \Arg{code case_n}
~ ~ \verb|}|
~ ~ \Arg{true code}
\end{syntax}
This function evaluates the \meta{test integer expression} and
compares this in turn to each of the
\meta{integer expression cases}. If the two are equal then the
associated \meta{code} is left in the input stream
and other cases are discarded. If any of the
cases are matched, the \meta{true code} is also inserted into the
input stream (after the code for the appropriate case).
\end{function}

\begin{function}{\IntCaseF}
\begin{syntax}
\cs{IntCaseF} \Arg{test integer expression}
~ ~ \verb|{|
~ ~ ~ ~ \Arg{intexpr case_1} \Arg{code case_1}
~ ~ ~ ~ \Arg{intexpr case_2} \Arg{code case_2}
~ ~ ~ ~ $\cdots$
~ ~ ~ ~ \Arg{intexpr case_n} \Arg{code case_n}
~ ~ \verb|}|
~ ~ \Arg{false code}
\end{syntax}
This function evaluates the \meta{test integer expression} and
compares this in turn to each of the
\meta{integer expression cases}. If the two are equal then the
associated \meta{code} is left in the input stream
and other cases are discarded. If none
match then the \meta{false code} is into the input stream 
(after the code for the appropriate case).
For example
\begin{demohigh}
\IgnoreSpacesOn
\IntCaseF { 2 * 5 }
  {
    { 5 }       { Small }
    { 4 + 6 }   { Medium }
    { -2 * 10 } { Negative }
  }
  { No idea! }
\IgnoreSpacesOff
\end{demohigh}
\end{function}

\begin{function}{\IntCaseTF}
\begin{syntax}
\cs{IntCaseTF} \Arg{test integer expression}
~ ~ \verb|{|
~ ~ ~ ~ \Arg{intexpr case_1} \Arg{code case_1}
~ ~ ~ ~ \Arg{intexpr case_2} \Arg{code case_2}
~ ~ ~ ~ $\cdots$
~ ~ ~ ~ \Arg{intexpr case_n} \Arg{code case_n}
~ ~ \verb|}|
~ ~ \Arg{true code}
~ ~ \Arg{false code}
\end{syntax}
This function evaluates the \meta{test integer expression} and
compares this in turn to each of the
\meta{integer expression cases}. If the two are equal then the
associated \meta{code} is left in the input stream
and other cases are discarded. If any of the
cases are matched, the \meta{true code} is also inserted into the
input stream (after the code for the appropriate case), while if none
match then the \meta{false code} is inserted.
\end{function}

\end{document}
