% -*- coding: utf-8 -*-
% !TEX program = lualatex
\documentclass[oneside]{book}

% -*- coding: utf-8 -*-
% !TEX program = lualatex

\usepackage[a4paper,margin=2.5cm]{geometry}

\newcommand*{\myversion}{2022F}
\newcommand*{\mydate}{Version \myversion\ (\the\year-\mylpad\month-\mylpad\day)}
\newcommand*{\mylpad}[1]{\ifnum#1<10 0\the#1\else\the#1\fi}

\setlength{\parindent}{0pt}
\setlength{\parskip}{4pt plus 1pt minus 1pt}

\usepackage{codehigh}

\colorlet{highback}{blue9}
%\CodeHigh{lite}
\CodeHigh{language=latex/latex2,style/main=highback,style/code=highback}
\NewCodeHighEnv{code}{style/main=gray9,style/code=gray9}
\NewCodeHighEnv{demo}{style/main=gray9,style/code=gray9,demo}

\usepackage{enumitem}

\NewDocumentCommand\MySubScript{m}{$_{#1}$}

\ExplSyntaxOn
\NewDocumentCommand\PrintVarList{m}{
  \clist_set:Nn \l_tmpa_clist {#1}
  \clist_map_inline:Nn \l_tmpa_clist
    {
      \token_to_str:N ##1 ~
    }
}
\NewDocumentCommand\RelaceChacters{m}{
  \tl_set:Nn \lTmpaTl {#1}
  \regex_replace_once:nnN { \_ } { \c{MySubScript} } \lTmpaTl
}
\NewDocumentCommand\RelaceUnderScoreAll{m}{
  \tl_set:Nn \lTmpaTl {#1}
  \regex_replace_all:nnN { \_ } { \c{_} } \lTmpaTl
  \lTmpaTl
}
\ExplSyntaxOff

\NewDocumentEnvironment{variable}{om}{
  \vspace{5pt}
  \begin{minipage}{\linewidth}
  \hrule\vspace{4pt}\obeylines%
  \begingroup
  \ttfamily\bfseries\color{azure3}
  \PrintVarList{#2}
  \endgroup
  \par\vspace{4pt}\hrule
  \end{minipage}\par\nopagebreak\vspace{4pt}
}{%
  \vspace{5pt}%
}

\NewDocumentEnvironment{function}{om}{
  \vspace{5pt}%
}{\vspace{5pt}}

\NewDocumentEnvironment{syntax}{}{%
  \begin{minipage}{\linewidth}
  \hrule\vspace{4pt}\obeylines%
}{%
  \par\vspace{4pt}\hrule
  \end{minipage}\par\nopagebreak\vspace{4pt}
}

\NewDocumentEnvironment{texnote}{}{}{}

\NewDocumentCommand\cs{O{}m}{%
  \texttt{\bfseries\color{purple3}\cBackslashStr\RelaceUnderScoreAll{#2}}%
}
\NewDocumentCommand\meta{m}{%
  \RelaceChacters{#1}%
  \textsl{$\langle$\ignorespaces\lTmpaTl\unskip$\rangle$}%
}
\NewDocumentCommand\Arg{m}{%
  \RelaceChacters{#1}%
  \texttt{\{}\textsl{$\langle$\ignorespaces\lTmpaTl\unskip$\rangle$}\texttt{\}}%
}

\NewDocumentCommand\pkg{m}{\textsf{#1}}

\NewDocumentCommand\nan{}{\texttt{NaN}}
\NewDocumentCommand\enquote{m}{``#1''}

\let\tn=\cs

\RenewDocumentCommand\emph{m}{%
  \underline{\textsl{#1}}%
}

\newenvironment{l3regex-syntax}
  {\begin{itemize}\def\\{\char`\\}\def\makelabel##1{\hss\llap{\ttfamily##1}}}
  {\end{itemize}}

\usepackage{shortvrb}

\NewDocumentCommand \MyMakeShortVerb {} {%
  \MakeShortVerb \|%
  \MakeShortVerb \"%
}
\NewDocumentCommand \MyDeleteShortVerb {} {%
  \DeleteShortVerb \|%
  \DeleteShortVerb \"%
}

\AtBeginDocument{\MyMakeShortVerb}
\AtEndDocument{\MyDeleteShortVerb}
\AddToHook{env/demohigh/before}{\MyDeleteShortVerb}
\AddToHook{env/demohigh/after}{\MyMakeShortVerb}

\usepackage{hologo}
\providecommand*\LuaTeX{\hologo{LuaTeX}}
\providecommand*\pdfTeX{\hologo{pdfTeX}}
\providecommand*\XeTeX{\hologo{XeTeX}}

\usepackage{amsmath}

\usepackage{hyperref}
\hypersetup{
  colorlinks=true,
  urlcolor=blue3,
  linkcolor=blue3,
}

\usepackage{functional}
%\Functional{scoping=false,tracing=true}

\CodeHigh{lite}

\begin{document}

\chapter{Integers (\texttt{Int})}

\section{Scratch Variables of Integers}

\begin{variable}{\lTmpaInt,\lTmpbInt,\lTmpcInt,\lTmpiInt,\lTmpjInt,\lTmpkInt}
Scratch integer for local assignment. These are never used by
the \verb!functional! package, and so are safe for use with any
function. However, they may be overwritten by other
code and so should only be used for short-term storage.
\end{variable}

\begin{variable}{\gTmpaInt,\gTmpbInt,\gTmpcInt,\gTmpiInt,\gTmpjInt,\gTmpkInt}
Scratch integer for global assignment. These are never used by
the \verb!functional! package, and so are safe for use with any
function. However, they may be overwritten by other
code and so should only be used for short-term storage.
\end{variable}

\section{Public Functions for Integers}

\begin{function}{\IntEval}
\begin{syntax}
\cs{IntEval} \Arg{integer expression}
\end{syntax}
Evaluates the \meta{integer expression} and returns the result:
for positive results an
explicit sequence of decimal digits not starting with~\texttt{0},
for negative results \texttt{-}~followed by such a sequence, and
\texttt{0}~for zero. For example
\begin{demohigh}
\IntEval{(1+4)*(2-3)/5}
\end{demohigh}
\end{function}

\begin{function}{\IntMathAdd}
\begin{syntax}
\cs{IntMathAdd} \Arg{integer expression_1} \Arg{integer expression_2}
\end{syntax}
Adds \Arg{integer expression_1} and \Arg{integer expression_2},
and returns the result. For example
\begin{demohigh}
\IntMathAdd{7}{3}
\end{demohigh}
\end{function}

\begin{function}{\IntMathSub}
\begin{syntax}
\cs{IntMathSub} \Arg{integer expression_1} \Arg{integer expression_2}
\end{syntax}
Subtracts \Arg{integer expression_2} from \Arg{integer expression_1},
and returns the result. For example
\begin{demohigh}
\IntMathSub{7}{3}
\end{demohigh}
\end{function}

\begin{function}{\IntMathMult}
\begin{syntax}
\cs{IntMathMult} \Arg{integer expression_1} \Arg{integer expression_2}
\end{syntax}
Multiplies \Arg{integer expression_1} by \Arg{integer expression_2},
and returns the result. For example
\begin{demohigh}
\IntMathMult{7}{3}
\end{demohigh}
\end{function}

\begin{function}{\IntMathDiv}
\begin{syntax}
\cs{IntMathDiv} \Arg{integer expression_1} \Arg{integer expression_2}
\end{syntax}
Divides \Arg{integer expression_1} by \Arg{integer expression_2},
and returns the result. For example
\begin{demohigh}
\IntMathDiv{7}{3}
\end{demohigh}
\end{function}

\begin{function}{\IntNew}
\begin{syntax}
\cs{IntNew} \meta{integer}
\end{syntax}
Creates a new \meta{integer} or raises an error if the name is
already taken. The declaration is global. The \meta{integer} is
initially equal to $0$.
\end{function}

\begin{function}{\IntUse}
\begin{syntax}
\cs{IntUse} \meta{integer}
\end{syntax}
Recovers the content of an \meta{integer} and returns the value.
An error is raised if the variable does not exist or if it is invalid.
\end{function}

\begin{function}{\IntSet}
\begin{syntax}
\cs{IntSet} \meta{integer} \Arg{integer expression}
\end{syntax}
Sets \meta{integer} to the value of \meta{integer expression},
which must evaluate to an integer (as described for \cs{IntEval}).
For example
\begin{demohigh}
\IntSet\lTmpaInt{3+5}
\IntUse\lTmpaInt
\end{demohigh}
\end{function}

\begin{function}{\IntZero}
\begin{syntax}
\cs{IntZero} \meta{integer}
\end{syntax}
Sets \meta{integer} to $0$. For example
\begin{demohigh}
\IntSet\lTmpaInt{5}
\IntZero\lTmpaInt
\IntUse\lTmpaInt
\end{demohigh}
\end{function}

\begin{function}{\IntIncr}
\begin{syntax}
\cs{IntIncr} \meta{integer}
\end{syntax}
Increases the value stored in \meta{integer} by $1$.
For example
\begin{demohigh}
\IntSet\lTmpaInt{5}
\IntIncr\lTmpaInt
\IntUse\lTmpaInt
\end{demohigh}
\end{function}

\begin{function}{\IntDecr}
\begin{syntax}
\cs{IntDecr} \meta{integer}
\end{syntax}
Decreases the value stored in \meta{integer} by $1$.
For example
\begin{demohigh}
\IntSet\lTmpaInt{5}
\IntDecr\lTmpaInt
\IntUse\lTmpaInt
\end{demohigh}
\end{function}

\begin{function}{\IntAdd}
\begin{syntax}
\cs{IntAdd} \meta{integer} \Arg{integer expression}
\end{syntax}
Adds the result of the \meta{integer expression} to the current
content of the \meta{integer}. For example
\begin{demohigh}
\IntSet\lTmpaInt{5}
\IntAdd\lTmpaInt{2}
\IntUse\lTmpaInt
\end{demohigh}
\end{function}

\begin{function}{\IntSub}
\begin{syntax}
\cs{IntSub} \meta{integer} \Arg{integer expression}
\end{syntax}
Subtracts the result of the \meta{integer expression} from the
current content of the \meta{integer}. For example
\begin{demohigh}
\IntSet\lTmpaInt{5}
\IntSub\lTmpaInt{3}
\IntUse\lTmpaInt
\end{demohigh}
\end{function}

\begin{function}{\IntStepVariable}
\begin{syntax}
\cs{IntStepVariable} \Arg{initial value} \Arg{step} \Arg{final value} \meta{tl~var} \Arg{code}
\end{syntax}
This function first evaluates the \meta{initial value}, \meta{step}
and \meta{final value}, all of which should be integer expressions.
Then for each \meta{value} from the \meta{initial value} to the
\meta{final value} in turn (using \meta{step} between each
\meta{value}), the \meta{code} is evaluated,
with the \meta{tl~var} defined as the current \meta{value}.  Thus
the \meta{code} should make use of the \meta{tl~var}.
For example
\begin{demohigh}
\IgnoreSpacesOn
\TlClear\lTmpaTl
\IntStepVariable{1}{3}{30}\lTmpiTl{
  \TlPutRight\lTmpaTl{\Value\lTmpiTl}
  \TlPutRight\lTmpaTl{~}
}
\Result{\Value\lTmpaTl}
\IgnoreSpacesOff
\end{demohigh}
\end{function}

\begin{function}{\IntCompare,\IntCompareTF}
\begin{syntax}
\cs{IntCompare} \Arg{intexpr_1} \meta{relation} \Arg{intexpr_2}
\cs{IntCompareTF} \Arg{intexpr_1} \meta{relation} \Arg{intexpr_2} \Arg{true code} \Arg{false code}
\end{syntax}
This function first evaluates each of the \meta{integer expressions}
as described for \cs{IntEval}. The two results are then
compared using the \meta{relation}:\par
{\centering\begin{tabular}{ll}
Equal        & \texttt{=} \\
Greater than & \texttt{>} \\
Less than    & \texttt{<} \\
\end{tabular}\par}
For example
\begin{demohigh}
\IntCompareTF{2}>{1}{\Result{Greater}}{\Result{Less}}
\IntCompareTF{2}>{3}{\Result{Greater}}{\Result{Less}}
\end{demohigh}
\end{function}

\end{document}
