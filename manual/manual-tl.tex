% -*- coding: utf-8 -*-
% !TEX program = lualatex
\documentclass[oneside]{book}

% -*- coding: utf-8 -*-
% !TEX program = lualatex

\usepackage[a4paper,margin=2.5cm]{geometry}

\newcommand*{\myversion}{2022F}
\newcommand*{\mydate}{Version \myversion\ (\the\year-\mylpad\month-\mylpad\day)}
\newcommand*{\mylpad}[1]{\ifnum#1<10 0\the#1\else\the#1\fi}

\setlength{\parindent}{0pt}
\setlength{\parskip}{4pt plus 1pt minus 1pt}

\usepackage{codehigh}

\colorlet{highback}{blue9}
%\CodeHigh{lite}
\CodeHigh{language=latex/latex2,style/main=highback,style/code=highback}
\NewCodeHighEnv{code}{style/main=gray9,style/code=gray9}
\NewCodeHighEnv{demo}{style/main=gray9,style/code=gray9,demo}

\usepackage{enumitem}

\NewDocumentCommand\MySubScript{m}{$_{#1}$}

\ExplSyntaxOn
\NewDocumentCommand\PrintVarList{m}{
  \clist_set:Nn \l_tmpa_clist {#1}
  \clist_map_inline:Nn \l_tmpa_clist
    {
      \token_to_str:N ##1 ~
    }
}
\NewDocumentCommand\RelaceChacters{m}{
  \tl_set:Nn \lTmpaTl {#1}
  \regex_replace_once:nnN { \_ } { \c{MySubScript} } \lTmpaTl
}
\NewDocumentCommand\RelaceUnderScoreAll{m}{
  \tl_set:Nn \lTmpaTl {#1}
  \regex_replace_all:nnN { \_ } { \c{_} } \lTmpaTl
  \lTmpaTl
}
\ExplSyntaxOff

\NewDocumentEnvironment{variable}{om}{
  \vspace{5pt}
  \begin{minipage}{\linewidth}
  \hrule\vspace{4pt}\obeylines%
  \begingroup
  \ttfamily\bfseries\color{azure3}
  \PrintVarList{#2}
  \endgroup
  \par\vspace{4pt}\hrule
  \end{minipage}\par\nopagebreak\vspace{4pt}
}{%
  \vspace{5pt}%
}

\NewDocumentEnvironment{function}{om}{
  \vspace{5pt}%
}{\vspace{5pt}}

\NewDocumentEnvironment{syntax}{}{%
  \begin{minipage}{\linewidth}
  \hrule\vspace{4pt}\obeylines%
}{%
  \par\vspace{4pt}\hrule
  \end{minipage}\par\nopagebreak\vspace{4pt}
}

\NewDocumentEnvironment{texnote}{}{}{}

\NewDocumentCommand\cs{O{}m}{%
  \texttt{\bfseries\color{purple3}\cBackslashStr\RelaceUnderScoreAll{#2}}%
}
\NewDocumentCommand\meta{m}{%
  \RelaceChacters{#1}%
  \textsl{$\langle$\ignorespaces\lTmpaTl\unskip$\rangle$}%
}
\NewDocumentCommand\Arg{m}{%
  \RelaceChacters{#1}%
  \texttt{\{}\textsl{$\langle$\ignorespaces\lTmpaTl\unskip$\rangle$}\texttt{\}}%
}

\NewDocumentCommand\pkg{m}{\textsf{#1}}

\NewDocumentCommand\nan{}{\texttt{NaN}}
\NewDocumentCommand\enquote{m}{``#1''}

\let\tn=\cs

\RenewDocumentCommand\emph{m}{%
  \underline{\textsl{#1}}%
}

\newenvironment{l3regex-syntax}
  {\begin{itemize}\def\\{\char`\\}\def\makelabel##1{\hss\llap{\ttfamily##1}}}
  {\end{itemize}}

\usepackage{shortvrb}

\NewDocumentCommand \MyMakeShortVerb {} {%
  \MakeShortVerb \|%
  \MakeShortVerb \"%
}
\NewDocumentCommand \MyDeleteShortVerb {} {%
  \DeleteShortVerb \|%
  \DeleteShortVerb \"%
}

\AtBeginDocument{\MyMakeShortVerb}
\AtEndDocument{\MyDeleteShortVerb}
\AddToHook{env/demohigh/before}{\MyDeleteShortVerb}
\AddToHook{env/demohigh/after}{\MyMakeShortVerb}

\usepackage{hologo}
\providecommand*\LuaTeX{\hologo{LuaTeX}}
\providecommand*\pdfTeX{\hologo{pdfTeX}}
\providecommand*\XeTeX{\hologo{XeTeX}}

\usepackage{amsmath}

\usepackage{hyperref}
\hypersetup{
  colorlinks=true,
  urlcolor=blue3,
  linkcolor=blue3,
}

\usepackage{functional}
%\Functional{scoping=false,tracing=true}

\CodeHigh{lite}

\begin{document}

\chapter{Token Lists (\texttt{Tl})}

\TeX{} works with tokens, and \LaTeX3 therefore provides a number of
functions to deal with lists of tokens.  Token lists may be present
directly in the argument to a function:
\begin{codehigh}
\TlFoo {a collection of \tokens}
\end{codehigh}
or may be stored in a so-called \enquote{token list variable}, which
have the suffix \texttt{Tl}: a token list variable can also be used as
the argument to a function, for example
\begin{codehigh}
\TlVarFoo \lSomeTl
\end{codehigh}
In both cases, functions are available to test and manipulate the lists
of tokens, and these have the module prefix \texttt{Tl}.
In many cases, functions which can be applied to token list variables
are paired with similar functions for application to explicit lists
of tokens: the two \enquote{views} of a token list are therefore collected
together here.

A token list (explicit, or stored in a variable) can be seen either
as a list of \enquote{items},
or a list of \enquote{tokens}. An item is whatever \cs{UseOne} would
grab as its argument: a single non-space token or a brace group,
with optional leading explicit space characters (each item is thus
itself a token list). A token is either a normal \texttt{N} argument,
or \verb*| |, \verb|{|, or \verb|}| (assuming normal \TeX{} category codes).
Thus for example
\begin{codehigh}
{Hello} world
\end{codehigh}
contains 6 items (\texttt{Hello}, \texttt{w}, \texttt{o}, \texttt{r},
\texttt{l} and \texttt{d}), but 13 tokens (\verb|{|, \texttt{H}, \texttt{e},
\texttt{l}, \texttt{l}, \texttt{o}, \verb|}|, \verb*| |, \texttt{w}, \texttt{o},
\texttt{r}, \texttt{l} and \texttt{d}).
Functions which act on items are often faster than their analogue acting
directly on tokens.

\section{Constant and Scratch Token Lists}

\begin{variable}{\cSpaceTl}
An explicit space character contained in a token list%
%(compare this with \cs{c_space_token})
. For use where an explicit space is required.
\end{variable}

\begin{variable}{\cEmptyTl}
Constant that is always empty.
\end{variable}

%\begin{variable}{\cNoValueTl}
%A marker for the absence of an argument. This constant \verb|tl| can safely
%be typeset (\emph{cf.}~\cs{qNil}), with the result being \verb|-NoValue-|.
%It is important to note that \cs{cNoValueTl} is constructed such that it
%will \emph{not} match the simple text input \verb|-NoValue-|, \emph{i.e.}
%that
%\begin{demohigh}
%\TlIfEqTF {\TlUse\cNoValueTl} {-NoValue-} {Result{Yes}} {\Result{No}}
%\end{demohigh}
%The \cs{cNoValueTl} marker is intended for
%use in creating document-level interfaces, where it serves as an indicator
%that an (optional) argument was omitted. In particular, it is distinct
%from a simple empty \verb|tl|.
%\end{variable}

\begin{variable}{\lTmpaTl,\lTmpbTl,\lTmpcTl,\lTmpiTl,\lTmpjTl,\lTmpkTl}
Scratch token lists for local assignment. These are never used by
the \verb!functional! package, and so are safe for use with any
function. However, they may be overwritten by other
code and so should only be used for short-term storage.
\end{variable}

\begin{variable}{\gTmpaTl,\gTmpbTl,\gTmpcTl,\gTmpiTl,\gTmpjTl,\gTmpkTl}
Scratch token lists for global assignment. These are never used by
the \verb!functional! package, and so are safe for use with any
function. However, they may be overwritten by other
code and so should only be used for short-term storage.
\end{variable}

\section{Creating and Using Token Lists}

\begin{function}{\TlNew}
\begin{syntax}
\cs{TlNew} \meta{tl var}
\end{syntax}
Creates a new \meta{tl var} or raises an error if the
name is already taken. The declaration is global. The
\meta{tl~var} is initially empty.
\begin{codehigh}
\TlNew \lFooSomeTl
\end{codehigh}
\end{function}

\begin{function}{\TlConst}
\begin{syntax}
\cs{TlConst} \meta{tl var} \Arg{token list}
\end{syntax}
Creates a new constant \meta{tl var} or raises an error
if the name is already taken. The value of the
\meta{tl var} is set globally to the \meta{token list}.
\begin{codehigh}
\TlConst \cFooSomeTl {abc}
\end{codehigh}
\end{function}

\begin{function}{\TlUse}
\begin{syntax}
\cs{TlUse} \meta{tl~var}
\end{syntax}
Recovers the content of a \meta{tl~var} and returns the value.
An error is raised if the variable
does not exist or if it is invalid. Note that it is possible to use
a \meta{tl~var} directly without an accessor function.
\begin{codehigh}
\TlUse \lTmpbTl
\end{codehigh}
\end{function}

\begin{function}{\TlToStr}
\begin{syntax}
\cs{TlToStr} \Arg{token list}
\end{syntax}
Converts the \meta{token list} to a \meta{string}, returning the
resulting character tokens. A \meta{string}
is a series of tokens with category code $12$ (other) with the exception
of spaces, which retain category code $10$ (space).
\begin{demohigh}
\TlToStr {12\abc34}
\end{demohigh}
%This function requires only a single expansion.
%Its argument \emph{must} be braced.
%\begin{texnote}
%This is the \eTeX{} primitive \tn{detokenize}.
%Converting a \meta{token list} to a \meta{string} yields a
%concatenation of the string representations of every token in the
%\meta{token list}.
%The string representation of a control sequence is
%\begin{itemize}
%\item an escape character, whose character code is given by the
%internal parameter \tn{escapechar}, absent if the
%\tn{escapechar} is negative or greater than the largest
%character code;
%\item the control sequence name, as defined by \cs{cs_to_str:N};
%\item a space, unless the control sequence name is a single
%character whose category at the time of expansion of
%\cs{tl_to_str:n} is not \enquote{letter}.
%\end{itemize}
%The string representation of an explicit character token is that
%character, doubled in the case of (explicit) macro parameter
%characters (normally \verb|#|).
%In particular, the string representation of a token list may
%depend on the category codes in effect when it is evaluated, and
%the value of the \tn{escapechar}: for instance |\tl_to_str:n {\a}|
%normally produces the three character \enquote{backslash},
%\enquote{lower-case a}, \enquote{space}, but it may also produce a
%single \enquote{lower-case a} if the escape character is negative
%and \texttt{a} is currently not a letter.
%\end{texnote}
\end{function}

\begin{function}{\TlVarToStr}
\begin{syntax}
\cs{TlVarToStr} \meta{tl var}
\end{syntax}
Converts the content of the \meta{tl var} to a string, returning the
resulting character tokens. A \meta{string}
is a series of tokens with category code $12$ (other) with the exception
of spaces, which retain category code $10$ (space).
\begin{demohigh}
\TlSet \lTmpaTl {12\abc34}
\TlVarToStr \lTmpaTl
\end{demohigh}
\end{function}

\section{Viewing Token Lists}

\begin{function}{\TlLog}
\begin{syntax}
\cs{TlLog} \Arg{token list}
\end{syntax}
Writes the \meta{token list} in the log file. See also
\cs{TlShow} which displays the result in the terminal.
\begin{codehigh}
\TlLog {123\abc456}
\end{codehigh}
\end{function}

\begin{function}{\TlVarLog}
\begin{syntax}
\cs{TlVarLog} \meta{tl var}
\end{syntax}
Writes the content of the \meta{tl var} in the log file. See also
\cs{TlVarShow} which displays the result in the terminal.
\begin{codehigh}
\TlSet \lTmpaTl {123\abc456}
\TlVarLog \lTmpaTl
\end{codehigh}
\end{function}

\begin{function}{\TlShow}
\begin{syntax}
\cs{TlShow} \Arg{token list}
\end{syntax}
Displays the \meta{token list} on the terminal.
\begin{codehigh}
\TlShow {123\abc456}
\end{codehigh}
%\begin{texnote}
%This is similar to the \eTeX{} primitive \tn{showtokens}, wrapped
%to a fixed number of characters per line.
%\end{texnote}
\end{function}

\begin{function}{\TlVarShow}
\begin{syntax}
\cs{TlVarShow} \meta{tl var}
\end{syntax}
Displays the content of the \meta{tl var} on the terminal.
\begin{codehigh}
\TlSet \lTmpaTl {123\abc456}
\TlVarShow \lTmpaTl
\end{codehigh}
%\begin{texnote}
%This is similar to the \TeX{} primitive \tn{show}, wrapped to a
%fixed number of characters per line.
%\end{texnote}
\end{function}

\section{Setting Token List Variables}

\begin{function}{\TlSet}
\begin{syntax}
\cs{TlSet} \meta{tl~var} \Arg{tokens}
\end{syntax}
Sets \meta{tl~var} to contain \meta{tokens},
removing any previous content from the variable.
\begin{demohigh}
\TlSet \lTmpiTl {\IntMathMult{4}{5}}
\TlUse \lTmpiTl
\end{demohigh}
\end{function}

\begin{function}{\TlSetEq}
\begin{syntax}
\cs{TlSetEq} \meta{tl var_1} \meta{tl var_2}
\end{syntax}
Sets the content of \meta{tl var_1} equal to that of \meta{tl var_2}.
\begin{demohigh}
\TlSet \lTmpaTl {abc}
\TlSetEq \lTmpbTl \lTmpaTl
\TlUse \lTmpbTl
\end{demohigh}
\end{function}

\begin{function}{\TlClear}
\begin{syntax}
\cs{TlClear} \meta{tl~var}
\end{syntax}
Clears all entries from the \meta{tl~var}.
\begin{demohigh}
\TlSet \lTmpjTl {One}
\TlClear \lTmpjTl
\TlSet \lTmpjTl {Two}
\TlUse \lTmpjTl
\end{demohigh}
\end{function}

\begin{function}{\TlClearNew}
\begin{syntax}
\cs{TlClearNew} \meta{tl var}
\end{syntax}
Ensures that the \meta{tl var} exists globally by applying
\cs{TlNew} if necessary, then applies \cs{TlClear} to leave
the \meta{tl var} empty.
\begin{codehigh}
\TlClearNew \lFooSomeTl
\end{codehigh}
\end{function}

\begin{function}{\TlConcat}
\begin{syntax}
\cs{TlConcat} \meta{tl var_1} \meta{tl var_2} \meta{tl var_3}
\end{syntax}
Concatenates the content of \meta{tl var_2} and \meta{tl var_3}
together and saves the result in \meta{tl var_1}. The \meta{tl var_2}
is placed at the left side of the new token list.
\begin{demohigh}
\TlSet \lTmpbTl {con}
\TlSet \lTmpcTl {cat}
\TlConcat \lTmpaTl \lTmpbTl \lTmpcTl
\TlUse \lTmpaTl
\end{demohigh}
\end{function}

\begin{function}{\TlPutLeft}
\begin{syntax}
\cs{TlPutLeft} \meta{tl~var} \Arg{tokens}
\end{syntax}
Appends \meta{tokens} to the left side of the current content of \meta{tl~var}.
\begin{demohigh}
\TlSet \lTmpkTl {Functional}
\TlPutLeft \lTmpkTl {Hello}
\TlUse \lTmpkTl
\end{demohigh}
\end{function}

\begin{function}{\TlPutRight}
\begin{syntax}
\cs{TlPutRight} \meta{tl~var} \Arg{tokens}
\end{syntax}
Appends \meta{tokens} to the right side of the current content of \meta{tl~var}.
\begin{demohigh}
\TlSet \lTmpkTl {Functional}
\TlPutRight \lTmpkTl {World}
\TlUse \lTmpkTl
\end{demohigh}
\end{function}

\section{Replacing Tokens}

Within token lists, replacement takes place at the top level: there is
no recursion into brace groups (more precisely, within a group defined by
a categroy code $1$/$2$ pair).

\begin{function}{\TlVarReplaceOnce}
\begin{syntax}
\cs{TlVarReplaceOnce} \meta{tl var} \Arg{old tokens} \Arg{new tokens}
\end{syntax}
Replaces the first (leftmost) occurrence of \meta{old tokens} in the
\meta{tl var} with \meta{new tokens}. \meta{Old tokens}
cannot contain \verb|{|, \verb|}| or \verb|#|
(more precisely, explicit character tokens with category code $1$
(begin-group) or $2$ (end-group), and tokens with category code $6$).
\begin{demohigh}
\TlSet \lTmpaTl {1{bc}2bc3}
\TlVarReplaceOnce \lTmpaTl {bc} {xx}
\TlUse \lTmpaTl
\end{demohigh}
\end{function}

\begin{function}{\TlVarReplaceAll}
\begin{syntax}
\cs{TlVarReplaceAll} \meta{tl var} \Arg{old tokens} \Arg{new tokens}
\end{syntax}
Replaces all occurrences of \meta{old tokens} in the
\meta{tl var} with \meta{new tokens}. \meta{Old tokens}
cannot contain \verb|{|, \verb|}| or \verb|#|
(more precisely, explicit character tokens with category code $1$
(begin-group) or $2$ (end-group), and tokens with category code $6$).
As this function
operates from left to right, the pattern \meta{old tokens}
may remain after the replacement (see \cs{TlVarRemoveAll} for an example).
\begin{demohigh}
\TlSet \lTmpaTl {1{bc}2bc3}
\TlVarReplaceAll \lTmpaTl {bc} {xx}
\TlUse \lTmpaTl
\end{demohigh}
\end{function}

\begin{function}{\TlVarRemoveOnce}
\begin{syntax}
\cs{TlVarRemoveOnce} \meta{tl var} \Arg{tokens}
\end{syntax}
Removes the first (leftmost) occurrence of \meta{tokens} from the
\meta{tl var}. \meta{Tokens} cannot contain \verb|{|, \verb|}| or \verb|#|
(more precisely, explicit character tokens with category code $1$
(begin-group) or $2$ (end-group), and tokens with category code $6$).
\begin{demohigh}
\TlSet \lTmpaTl {1{bc}2bc3}
\TlVarRemoveOnce \lTmpaTl {bc}
\TlUse \lTmpaTl
\end{demohigh}
\end{function}

\begin{function}{\TlVarRemoveAll}
\begin{syntax}
\cs{TlVarRemoveAll} \meta{tl var} \Arg{tokens}
\end{syntax}
Removes all occurrences of \meta{tokens} from the
\meta{tl var}. \meta{Tokens} cannot contain \verb|{|, \verb|}| or \verb|#|
(more precisely, explicit character tokens with category code $1$
(begin-group) or $2$ (end-group), and tokens with category code $6$).
As this function
operates from left to right, the pattern \meta{tokens}
may remain after the removal, for instance,
\begin{demohigh}
\TlSet \lTmpaTl {abbccd}
\TlVarRemoveAll \lTmpaTl {bc}
\TlUse \lTmpaTl
\end{demohigh}
\end{function}

\begin{function}{\TlTrimSpaces}
\begin{syntax}
\cs{TlTrimSpaces} \Arg{token list}
\end{syntax}
Removes any leading and trailing explicit space characters
(explicit tokens with character code $32$ and category code $10$)
from the \meta{token list} and returns the result.
\begin{demohigh}
Foo\TlTrimSpaces { 12 34 }Bar
\end{demohigh}
%\begin{texnote}
%The result is returned within \tn{unexpanded}, which means that the token
%list does not expand further when appearing in an \texttt{x}-type
%or \texttt{e}-type argument expansion.
%\end{texnote}
\end{function}

\begin{function}{\TlVarTrimSpaces}
\begin{syntax}
\cs{TlVarTrimSpaces} \meta{tl var}
\end{syntax}
Sets the \meta{tl var} to contain the result of removing any leading
and trailing explicit space characters (explicit tokens with
character code $32$ and category code $10$) from its contents.
\begin{demohigh}
\TlSet \lTmpaTl { 12 34 }
\TlVarTrimSpaces \lTmpaTl
Foo\TlUse \lTmpaTl Bar
\end{demohigh}
\end{function}

\section{Working with the Content of Token Lists}

\begin{function}{\TlCount}
\begin{syntax}
\cs{TlCount} \Arg{tokens}
\end{syntax}
Counts the number of \meta{items} in \meta{tokens} and returns this information.
Unbraced tokens count as one element as do each token group (\verb|{|$\cdots$\verb|}|).
This process ignores any unprotected spaces within \meta{tokens}. %See also \cs{TlVarCount}.
%This function requires three expansions, giving an \meta{integer denotation}.
\begin{demohigh}
\TlCount {12\abc34}
\end{demohigh}
\end{function}

\begin{function}{\TlVarCount}
\begin{syntax}
\cs{TlVarCount} \meta{tl var}
\end{syntax}
Counts the number of \meta{items} in the \meta{tl var} and returns this information.
Unbraced tokens count as one element as do each token group (\verb|{|$\cdots$\verb|}|).
This process ignores any unprotected spaces within the \meta{tl var}. %See also \cs{TlCount}.
%This function requires three expansions, giving an \meta{integer denotation}.
\begin{demohigh}
\TlSet \lTmpaTl {12\abc34}
\TlVarCount \lTmpaTl
\end{demohigh}
\end{function}

\begin{function}{\TlHead}
\begin{syntax}
\cs{TlHead} \Arg{token list}
\end{syntax}
Returns the first \meta{item} in the \meta{token list},
discarding the rest of the \meta{token list}.
All leading explicit space characters
(explicit tokens with character code $32$ and category code $10$)
are discarded; for example
\begin{demohigh}
\fbox {1\TlHead{ abc }2}
\fbox {1\TlHead{  abc }2}
\end{demohigh}
If the \enquote{head} is a brace group, rather than a single token,
the braces are removed, and so
\begin{codehigh}
\TlHead { { ab} c }
\end{codehigh}
yields \verb*| ab|.
A blank \meta{token list} (see \cs{TlIfBlank}) results in
\cs{TlHead} returning nothing.
%\begin{texnote}
%The result is returned within \cs{exp_not:n}, which means that the token
%list does not expand further when appearing in an \texttt{x}-type
%or \texttt{e}-type argument expansion.
%\end{texnote}
\end{function}

\begin{function}{\TlVarHead}
\begin{syntax}
\cs{TlVarHead} \meta{tl var}
\end{syntax}
Returns the first \meta{item} in the \meta{tl var},
discarding the rest of the \meta{tl var}.
All leading explicit space characters (explicit tokens with character code $32$
and category code $10$) are discarded.
\begin{demohigh}
\TlSet \lTmpaTl {HELLO}
\TlVarHead \lTmpaTl
\end{demohigh}
\end{function}

\begin{function}{\TlTail}
\begin{syntax}
\cs{TlTail} \Arg{token list}
\end{syntax}
Discards all leading explicit space characters
(explicit tokens with character code $32$ and category code $10$)
and the first \meta{item} in the \meta{token list}, and returns the
remaining tokens. Thus for example
\begin{codehigh}
\TlTail { a {bc} d }
\end{codehigh}
and
\begin{codehigh}
\TlTail {  a {bc} d }
\end{codehigh}
both return \verb*| {bc} d |.  A blank \meta{token list} (see \cs{TlIfBlank})
results in \cs{TlTail} returning nothing.
%\begin{texnote}
%The result is returned within \cs{exp_not:n}, which means that the
%token list does not expand further when appearing in an \texttt{x}-type
%or \texttt{e}-type argument expansion.
%\end{texnote}
\end{function}

\begin{function}{\TlVarTail}
\begin{syntax}
\cs{TlVarTail} \meta{tl var}
\end{syntax}
Discards all leading explicit space characters
(explicit tokens with character code $32$ and category code $10$)
and the first \meta{item} in the \meta{tl var}, and returns the
remaining tokens.
\begin{demohigh}
\TlSet \lTmpaTl {HELLO}
\TlVarTail \lTmpaTl
\end{demohigh}
\end{function}

\begin{function}{\TlItem,\TlVarItem}
\begin{syntax}
\cs{TlItem} \Arg{token list} \Arg{integer expression}
\cs{TlVarItem} \meta{tl var} \Arg{integer expression}
\end{syntax}
Indexing items in the \meta{token list} from $1$ on the left, this
function evaluates the \meta{integer expression} and returns the
appropriate item from the \meta{token list}.
If the \meta{integer expression} is negative, indexing occurs from
the right of the token list, starting at $-1$ for the right-most item.
If the index is out of bounds, then the function returns nothing.
\begin{demohigh}
\TlItem {abcd} {3}
\end{demohigh}
%\begin{texnote}
%The result is returned within the \tn{unexpanded}
%primitive (\cs{exp_not:n}), which means that the \meta{item}
%does not expand further when appearing in an \texttt{x}-type
%or \texttt{e}-type argument expansion.
%\end{texnote}
\end{function}

\begin{function}{\TlRandItem,\TlVarRandItem}
\begin{syntax}
\cs{TlRandItem} \Arg{token list}
\cs{TlVarRandItem} \meta{tl var}
\end{syntax}
Selects and returns a pseudo-random item of the \meta{token list}.
If the \meta{token list} is blank, the result is empty.
%This is not available in older versions of \XeTeX{}.
\begin{demohigh}
\TlRandItem {abcdef}
\TlRandItem {abcdef}
\end{demohigh}
%\begin{texnote}
%The result is returned within the \tn{unexpanded}
%primitive (\cs{exp_not:n}), which means that the \meta{item}
%does not expand further when appearing in an \texttt{x}-type
%or \texttt{e}-type argument expansion.
%\end{texnote}
\end{function}

\section{Mapping over Token Lists}

All mappings are done at the current group level, \emph{i.e.} any
local assignments made by the \meta{function} or \meta{code} discussed
below remain in effect after the loop.

%\begin{function}{\TlMapFunction}
%\begin{syntax}
%\cs{TlMapFunction} \Arg{token list} \meta{function}
%\end{syntax}
%Applies \meta{function} to every \meta{item} in the \meta{token list},
%The \meta{function} receives one argument for each iteration.
%This may be a number of tokens if the \meta{item} was stored within
%braces. Hence the \meta{function} should anticipate receiving
%\texttt{n}-type arguments.
%\end{function}
%
%\begin{function}{\TlVarMapFunction}
%\begin{syntax}
%\cs{TlVarMapFunction} \meta{tl var} \meta{function}
%\end{syntax}
%Applies \meta{function} to every \meta{item} in the \meta{tl var}.
%The \meta{function} receives one argument for each iteration.
%This may be a number of tokens if the \meta{item} was stored within
%braces. Hence the \meta{function} should anticipate receiving
%\texttt{n}-type arguments.
%\end{function}

\begin{function}{\TlMapInline}
\begin{syntax}
\cs{TlMapInline} \Arg{token list} \Arg{inline function}
\end{syntax}
Applies the \meta{inline function} to every \meta{item} stored within the
\meta{token list}. The \meta{inline function}  should consist of code which
receives the \meta{item} as \verb|#1|. For example,
\begin{codehigh}
\IgnoreSpacesOn
\TlClear \lTmpaTl
\TlMapInline {one} {
  \TlPutRight \lTmpaTl {[#1]}
}
\Result{\TlUse\lTmpaTl}
\IgnoreSpacesOff
\end{codehigh}
produces [o][n][e].
\end{function}

\begin{function}{\TlVarMapInline}
\begin{syntax}
\cs{TlVarMapInline} \meta{tl var} \Arg{inline function}
\end{syntax}
Applies the \meta{inline function} to every \meta{item} stored within the
\meta{tl var}. The \meta{inline function} should consist of code which
receives the \meta{item} as \verb|#1|. For example,
\begin{codehigh}
\IgnoreSpacesOn
\TlClear \lTmpaTl
\TlSet \lTmpkTl {one}
\TlVarMapInline \lTmpkTl {
  \TlPutRight \lTmpaTl {[#1]}
}
\Result{\TlUse\lTmpaTl}
\IgnoreSpacesOff
\end{codehigh}
produces [o][n][e].
\end{function}

%\begin{function}{\TlMapTokens,\TlVarMapTokens}
%\begin{syntax}
%\cs{TlMapTokens} \Arg{tokens} \Arg{code}
%\cs{TlVarMapTokens} \meta{tl var} \Arg{code}
%\end{syntax}
%Analogue of \cs{tl_map_function:NN} which maps several tokens
%instead of a single function.  The \meta{code} receives each \meta{item} in
%the \meta{tl var} or in \meta{tokens} as a trailing brace group. For
%instance,
%\begin{verbatim}
%\tl_map_tokens:Nn \l_my_tl { \prg_replicate:nn { 2 } }
%\end{verbatim}
%expands to twice each \meta{item} in the \meta{tl var}: for each \meta{item} in
%\cs{l_my_tl} the function \cs{prg_replicate:nn} receives |2| and
%\meta{item} as its two arguments.  The function
%\cs{tl_map_inline:Nn} is typically faster but is not expandable.
%\end{function}

\begin{function}{\TlMapVariable}
\begin{syntax}
\cs{TlMapVariable} \Arg{token list} \meta{variable} \Arg{code}
\end{syntax}
Stores each \meta{item} of the \meta{token list} in turn in the
(token list) \meta{variable} and applies the \meta{code}.  The
\meta{code} will usually make use of the \meta{variable}, but this
is not enforced.  The assignments to the \meta{variable} are local.
Its value after the loop is the last \meta{item} in the
\meta{tl var}, or its original value if the \meta{tl var} is blank.
\begin{demohigh}
\IgnoreSpacesOn
\TlClear \lTmpaTl
\TlMapVariable {one} \lTmpiTl {
  \TlPutRight \lTmpaTl {\Expand {[\lTmpiTl]}}
}
\Result{\TlUse\lTmpaTl}
\IgnoreSpacesOff
\end{demohigh}
\end{function}

\begin{function}{\TlVarMapVariable}
\begin{syntax}
\cs{TlVarMapVariable} \meta{tl var} \meta{variable} \Arg{code}
\end{syntax}
Stores each \meta{item} of the \meta{tl var} in turn in the (token
list) \meta{variable} and applies the \meta{code}.  The \meta{code}
will usually make use of the \meta{variable}, but this is not
enforced.  The assignments to the \meta{variable} are local.  Its
value after the loop is the last \meta{item} in the \meta{tl var},
or its original value if the \meta{tl var} is blank.
\begin{demohigh}
\IgnoreSpacesOn
\TlClear \lTmpaTl
\TlSet \lTmpkTl {one}
\TlVarMapVariable \lTmpkTl \lTmpiTl {
  \TlPutRight \lTmpaTl {\Expand {[\lTmpiTl]}}
}
\Result{\TlUse\lTmpaTl}
\IgnoreSpacesOff
\end{demohigh}
\end{function}

%\begin{function}{\TlMapBreak}
%\begin{syntax}
%\cs{TlMapBreak}
%\end{syntax}
%Used to terminate a tl map function before all
%entries in the \meta{token list variable} have been processed. This
%normally takes place within a conditional statement, for example
%\begin{verbatim}
%\tl_map_inline:Nn \l_my_tl
%{
%\str_if_eq:nnT { #1 } { bingo } { \tl_map_break: }
%Do something useful
%}
%\end{verbatim}
%See also \cs{tl_map_break:n}.
%Use outside of a tl map scenario leads to low
%level \TeX{} errors.
%\begin{texnote}
%When the mapping is broken, additional tokens may be inserted
%before the \meta{tokens} are
%inserted into the input stream.
%This depends on the design of the mapping function.
%\end{texnote}
%\end{function}
%
%\begin{function}{\TlMapBreakDo}
%\begin{syntax}
%\cs{TlMapBreakDo} \Arg{code}
%\end{syntax}
%Used to terminate a tl map function before all
%entries in the \meta{token list variable} have been processed, inserting
%the \meta{code} after the mapping has ended. This
%normally takes place within a conditional statement, for example
%\begin{verbatim}
%\tl_map_inline:Nn \l_my_tl
%{
%\str_if_eq:nnT { #1 } { bingo }
%{ \tl_map_break:n { <code> } }
%Do something useful
%}
%\end{verbatim}
%Use outside of a tl map scenario leads to low
%level \TeX{} errors.
%\begin{texnote}
%When the mapping is broken, additional tokens may be inserted
%before the \meta{code} is
%inserted into the input stream.
%This depends on the design of the mapping function.
%\end{texnote}
%\end{function}

\section{Token List Conditionals}

\begin{function}{\TlIfExist,\TlIfExistTF}
\begin{syntax}
\cs{TlIfExist} \meta{tl var}
\cs{TlIfExistTF} \meta{tl var} \Arg{true code} \Arg{false code}
\end{syntax}
Tests whether the \meta{tl var} is currently defined.  This does not
check that the \meta{tl var} really is a token list variable.
\begin{demohigh}
\TlIfExistTF \lTmpaTl {\Result{Yes}} {\Result{No}}
\TlIfExistTF \lFooUndefinedTl {\Result{Yes}} {\Result{No}}
\end{demohigh}
\end{function}

\begin{function}{\TlIfEmpty,\TlIfEmptyTF}
\begin{syntax}
\cs{TlIfEmpty} \Arg{token list}
\cs{TlIfEmptyTF} \Arg{token list} \Arg{true code} \Arg{false code}
\end{syntax}
Tests if the \meta{token list} is entirely empty
(\emph{i.e.}~contains no tokens at all). For example
\begin{demohigh}
\TlIfEmptyTF {abc} {\Result{Empty}} {\Result{NonEmpty}}
\TlIfEmptyTF {} {\Result{Empty}} {\Result{NonEmpty}}
\end{demohigh}
\end{function}

\begin{function}{\TlVarIfEmpty,\TlVarIfEmptyTF}
\begin{syntax}
\cs{TlVarIfEmpty} \meta{tl~var}
\cs{TlVarIfEmptyTF} \meta{tl~var} \Arg{true code} \Arg{false code}
\end{syntax}
Tests if the \meta{token list variable} is entirely empty
(\emph{i.e.}~contains no tokens at all). For example
\begin{demohigh}
\TlSet \lTmpaTl {abc}
\TlVarIfEmptyTF \lTmpaTl {\Result{Empty}} {\Result{NonEmpty}}
\TlClear \lTmpaTl
\TlVarIfEmptyTF \lTmpaTl {\Result{Empty}} {\Result{NonEmpty}}
\end{demohigh}
\end{function}

\begin{function}{\TlIfBlank}
\begin{syntax}
\cs{TlIfBlank} \Arg{token list}
\cs{TlIfBlankTF} \Arg{token list} \Arg{true code} \Arg{false code}
\end{syntax}
Tests if the \meta{token list} consists only of blank spaces
(\emph{i.e.} contains no item). The test is \texttt{true} if
\meta{token list} is zero or more explicit space characters
(explicit tokens with character code $32$ and category code $10$),
and is \texttt{false} otherwise.
\begin{demohigh}
\TlIfEmptyTF {  } {\Result{Yes}} {\Result{No}}
\TlIfBlankTF {  } {\Result{Yes}} {\Result{No}}
\end{demohigh}
\end{function}

\begin{function}{\TlIfEq,\TlIfEqTF}
\begin{syntax}
\cs{TlIfEq} \Arg{token list_1} \Arg{token list_2}
\cs{TlIfEqTF} \Arg{token list_1} \Arg{token list_2} \Arg{true code} \Arg{false code}
\end{syntax}
Tests if \meta{token list_1} and \meta{token list_2} contain the
same list of tokens, both in respect of character codes and category
codes. See \cs{StrIfEq} if category codes are not important.
For example
\begin{demohigh}
\TlIfEqTF {abc} {abc} {\Result{Yes}} {\Result{No}}
\TlIfEqTF {abc} {xyz} {\Result{Yes}} {\Result{No}}
\end{demohigh}
%\begin{demohigh}
%\TlSet\lTmpaTl{abc}
%\TlSet\lTmpbTl{abc}
%\TlSet\lTmpcTl{xyz}
%\TlIfEqTF{\TlUse\lTmpaTl}{\TlUse\lTmpbTl}{\Result{Yes}}{\Result{No}}
%\TlIfEqTF{\TlUse\lTmpaTl}{\TlUse\lTmpcTl}{\Result{Yes}}{\Result{No}}
%\end{demohigh}
\end{function}

\begin{function}{\TlVarIfEq,\TlVarIfEqTF}
\begin{syntax}
\cs{TlVarIfEq} \meta{tl var_1} \meta{tl var_2}
\cs{TlVarIfEqTF} \meta{tl var_1} \meta{tl var_2} \Arg{true code} \Arg{false code}
\end{syntax}
Compares the content of two \meta{token list variables} and
is logically \texttt{true} if the two contain the same list of
tokens (\emph{i.e.}~identical in both the list of characters they
contain and the category codes of those characters). For example
\begin{demohigh}
\TlSet \lTmpaTl {abc}
\TlSet \lTmpbTl {abc}
\TlSet \lTmpcTl {xyz}
\TlVarIfEqTF \lTmpaTl \lTmpbTl {\Result{Yes}} {\Result{No}}
\TlVarIfEqTF \lTmpaTl \lTmpcTl {\Result{Yes}} {\Result{No}}
\end{demohigh}
See also \cs{StrVarIfEq} for a comparison that ignores category codes.
\end{function}

\begin{function}{\TlIfIn,\TlIfInTF}
\begin{syntax}
\cs{TlIfIn} \Arg{token list_1} \Arg{token list_2}
\cs{TlIfInTF} \Arg{token list_1} \Arg{token list_2} \Arg{true code} \Arg{false code}
\end{syntax}
Tests if \meta{token list_2} is found inside \meta{token list_1}.
The \meta{token list_2} cannot contain the tokens \verb|{|, \verb|}| or \verb|#|
(more precisely, explicit character tokens with category code $1$
(begin-group) or $2$ (end-group), and tokens with category code $6$).
The search does \emph{not} enter brace (category code $1$/$2$) groups.
\begin{demohigh}
\TlIfInTF {hello world} {o} {\Result{Yes}} {\Result{No}}
\TlIfInTF {hello world} {a} {\Result{Yes}} {\Result{No}}
\end{demohigh}
\end{function}

\begin{function}{\TlVarIfIn,\TlVarIfInTF}
\begin{syntax}
\cs{TlVarIfIn} \meta{tl var} \Arg{token list}
\cs{TlVarIfInTF} \meta{tl var} \Arg{token list} \Arg{true code} \Arg{false code}
\end{syntax}
Tests if the \meta{token list} is found in the content of the
\meta{tl var}. The \meta{token list} cannot contain
the tokens \verb|{|, \verb|}| or \verb|#|
(more precisely, explicit character tokens with category code $1$
(begin-group) or $2$ (end-group), and tokens with category code $6$).
\begin{demohigh}
\TlSet \lTmpaTl {hello world}
\TlVarIfInTF \lTmpaTl {o} {\Result{Yes}} {\Result{No}}
\TlVarIfInTF \lTmpaTl {a} {\Result{Yes}} {\Result{No}}
\end{demohigh}
\end{function}

\begin{function}{\TlIfSingle,\TlIfSingleTF}
\begin{syntax}
\cs{TlIfSingle} \Arg{token list}
\cs{TlIfSingleTF} \Arg{token list} \Arg{true code} \Arg{false code}
\end{syntax}
Tests if the \meta{token list} has exactly one \meta{item}, \emph{i.e.} is
a single normal token (neither an explicit space character nor a
begin-group character) or a single brace group, surrounded by
optional spaces on both sides. In other words, such a token list has
token count $1$ according to \cs{TlCount}.
\begin{demohigh}
\TlIfSingleTF {a} {\Result{Yes}} {\Result{No}}
\TlIfSingleTF {abc} {\Result{Yes}} {\Result{No}}
\end{demohigh}
\end{function}

\begin{function}{\TlVarIfSingle,\TlVarIfSingleTF}
\begin{syntax}
\cs{TlVarIfSingle} \meta{tl var}
\cs{TlVarIfSingleTF} \meta{tl var} \Arg{true code} \Arg{false code}
\end{syntax}
Tests if the content of the \meta{tl var} consists of a single \meta{item},
\emph{i.e.} is a single normal token (neither an explicit space
character nor a begin-group character) or a single brace group,
surrounded by optional spaces on both sides. In other words, such a
token list has token count $1$ according to \cs{TlVarCount}.
\begin{demohigh}
\TlSet \lTmpaTl {a}
\TlVarIfSingleTF \lTmpaTl {\Result{Yes}} {\Result{No}}
\TlSet \lTmpaTl {abc}
\TlVarIfSingleTF \lTmpaTl {\Result{Yes}} {\Result{No}}
\end{demohigh}
\end{function}

\section{Token List Case Functions}

\begin{function}{\TlVarCase}
\begin{syntax}
\cs{TlVarCase} \meta{test token list variable}
~ ~ \verb"{"
~ ~ ~ ~ \meta{token list variable case_1} \Arg{code case_1}
~ ~ ~ ~ \meta{token list variable case_2} \Arg{code case_2}
~ ~ ~ ~ $\cdots$
~ ~ ~ ~ \meta{token list variable case_n} \Arg{code case_n}
~ ~ \verb"}"
\end{syntax}
This function compares the \meta{test token list variable} in turn
with each of the \meta{token list variable cases}. If the two
are equal (as described for \cs{TlVarIfEq})
then the associated \meta{code} is left in the input
stream and other cases are discarded. The function
does nothing if there is no match.
\begin{demohigh}
\IgnoreSpacesOn
\TlSet \lTmpaTl {a}
\TlSet \lTmpbTl {b}
\TlSet \lTmpcTl {c}
\TlSet \lTmpkTl {b}
\TlVarCase \lTmpkTl {
  \lTmpaTl {\Result {First}}
  \lTmpbTl {\Result {Second}}
  \lTmpcTl {\Result {Third}}
}
\IgnoreSpacesOff
\end{demohigh}
\end{function}

\begin{function}{\TlVarCaseT}
\begin{syntax}
\cs{TlVarCaseT} \meta{test token list variable}
~ ~ \verb"{"
~ ~ ~ ~ \meta{token list variable case_1} \Arg{code case_1}
~ ~ ~ ~ \meta{token list variable case_2} \Arg{code case_2}
~ ~ ~ ~ $\cdots$
~ ~ ~ ~ \meta{token list variable case_n} \Arg{code case_n}
~ ~ \verb"}"
~ ~ \Arg{true code}
\end{syntax}
This function compares the \meta{test token list variable} in turn
with each of the \meta{token list variable cases}. If the two
are equal (as described for \cs{TlVarIfEq})
then the associated \meta{code} is left in the input
stream and other cases are discarded. If any of the
cases are matched, the \meta{true code} is also inserted into the
input stream (after the code for the appropriate case).
\begin{demohigh}
\IgnoreSpacesOn
\TlSet \lTmpaTl {a}
\TlSet \lTmpbTl {b}
\TlSet \lTmpcTl {c}
\TlSet \lTmpkTl {b}
\TlVarCaseT \lTmpkTl {
  \lTmpaTl {\IntSet \lTmpkInt {1}}
  \lTmpbTl {\IntSet \lTmpkInt {2}}
  \lTmpcTl {\IntSet \lTmpkInt {3}}
}{
  \Result {\IntUse \lTmpkInt}
}
\IgnoreSpacesOff
\end{demohigh}
\end{function}

\begin{function}{\TlVarCaseF}
\begin{syntax}
\cs{TlVarCaseF} \meta{test token list variable}
~ ~ \verb"{"
~ ~ ~ ~ \meta{token list variable case_1} \Arg{code case_1}
~ ~ ~ ~ \meta{token list variable case_2} \Arg{code case_2}
~ ~ ~ ~ $\cdots$
~ ~ ~ ~ \meta{token list variable case_n} \Arg{code case_n}
~ ~ \verb"}"
~ ~ \Arg{false code}
\end{syntax}
This function compares the \meta{test token list variable} in turn
with each of the \meta{token list variable cases}. If the two
are equal (as described for \cs{TlVarIfEq})
then the associated \meta{code} is left in the input
stream and other cases are discarded. If none match then the \meta{false code}
is inserted into the input stream (after the code for the appropriate case).
\begin{demohigh}
\IgnoreSpacesOn
\TlSet \lTmpaTl {a}
\TlSet \lTmpbTl {b}
\TlSet \lTmpcTl {c}
\TlSet \lTmpkTl {b}
\TlVarCaseF \lTmpkTl{
  \lTmpaTl {\Result {First}}
  \lTmpbTl {\Result {Second}}
  \lTmpcTl {\Result {Third}}
}{
  \Result {No~Match!}
}
\IgnoreSpacesOff
\end{demohigh}
\end{function}

\begin{function}{\TlVarCaseTF}
\begin{syntax}
\cs{TlVarCaseTF} \meta{test token list variable}
~ ~ \verb"{"
~ ~ ~ ~ \meta{token list variable case_1} \Arg{code case_1}
~ ~ ~ ~ \meta{token list variable case_2} \Arg{code case_2}
~ ~ ~ ~ $\cdots$
~ ~ ~ ~ \meta{token list variable case_n} \Arg{code case_n}
~ ~ \verb"}"
~ ~ \Arg{true code}
~ ~ \Arg{false code}
\end{syntax}
This function compares the \meta{test token list variable} in turn
with each of the \meta{token list variable cases}. If the two
are equal (as described for \cs{TlVarIfEq})
then the associated \meta{code} is left in the input
stream and other cases are discarded. If any of the
cases are matched, the \meta{true code} is also inserted into the
input stream (after the code for the appropriate case), while if none
match then the \meta{false code} is inserted. The function \cs{TlVarCase},
which does nothing if there is no match, is also available.
\begin{demohigh}
\IgnoreSpacesOn
\TlSet \lTmpaTl {a}
\TlSet \lTmpbTl {b}
\TlSet \lTmpcTl {c}
\TlSet \lTmpkTl {b}
\TlVarCaseTF \lTmpkTl {
  \lTmpaTl {\IntSet \lTmpkInt {1}}
  \lTmpbTl {\IntSet \lTmpkInt {2}}
  \lTmpcTl {\IntSet \lTmpkInt {3}}
}{
  \Result {\IntUse \lTmpkInt}
}{
  \Result {0}
}
\IgnoreSpacesOff
\end{demohigh}
\end{function}

\end{document}
