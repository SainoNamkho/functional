% -*- coding: utf-8 -*-
% !TEX program = lualatex
\documentclass[oneside]{book}

% -*- coding: utf-8 -*-
% !TEX program = lualatex

\usepackage[a4paper,margin=2.5cm]{geometry}

\newcommand*{\myversion}{2022F}
\newcommand*{\mydate}{Version \myversion\ (\the\year-\mylpad\month-\mylpad\day)}
\newcommand*{\mylpad}[1]{\ifnum#1<10 0\the#1\else\the#1\fi}

\setlength{\parindent}{0pt}
\setlength{\parskip}{4pt plus 1pt minus 1pt}

\usepackage{codehigh}

\colorlet{highback}{blue9}
%\CodeHigh{lite}
\CodeHigh{language=latex/latex2,style/main=highback,style/code=highback}
\NewCodeHighEnv{code}{style/main=gray9,style/code=gray9}
\NewCodeHighEnv{demo}{style/main=gray9,style/code=gray9,demo}

\usepackage{enumitem}

\NewDocumentCommand\MySubScript{m}{$_{#1}$}

\ExplSyntaxOn
\NewDocumentCommand\PrintVarList{m}{
  \clist_set:Nn \l_tmpa_clist {#1}
  \clist_map_inline:Nn \l_tmpa_clist
    {
      \token_to_str:N ##1 ~
    }
}
\NewDocumentCommand\RelaceChacters{m}{
  \tl_set:Nn \lTmpaTl {#1}
  \regex_replace_once:nnN { \_ } { \c{MySubScript} } \lTmpaTl
}
\NewDocumentCommand\RelaceUnderScoreAll{m}{
  \tl_set:Nn \lTmpaTl {#1}
  \regex_replace_all:nnN { \_ } { \c{_} } \lTmpaTl
  \lTmpaTl
}
\ExplSyntaxOff

\NewDocumentEnvironment{variable}{om}{
  \vspace{5pt}
  \begin{minipage}{\linewidth}
  \hrule\vspace{4pt}\obeylines%
  \begingroup
  \ttfamily\bfseries\color{azure3}
  \PrintVarList{#2}
  \endgroup
  \par\vspace{4pt}\hrule
  \end{minipage}\par\nopagebreak\vspace{4pt}
}{%
  \vspace{5pt}%
}

\NewDocumentEnvironment{function}{om}{
  \vspace{5pt}%
}{\vspace{5pt}}

\NewDocumentEnvironment{syntax}{}{%
  \begin{minipage}{\linewidth}
  \hrule\vspace{4pt}\obeylines%
}{%
  \par\vspace{4pt}\hrule
  \end{minipage}\par\nopagebreak\vspace{4pt}
}

\NewDocumentEnvironment{texnote}{}{}{}

\NewDocumentCommand\cs{O{}m}{%
  \texttt{\bfseries\color{purple3}\cBackslashStr\RelaceUnderScoreAll{#2}}%
}
\NewDocumentCommand\meta{m}{%
  \RelaceChacters{#1}%
  \textsl{$\langle$\ignorespaces\lTmpaTl\unskip$\rangle$}%
}
\NewDocumentCommand\Arg{m}{%
  \RelaceChacters{#1}%
  \texttt{\{}\textsl{$\langle$\ignorespaces\lTmpaTl\unskip$\rangle$}\texttt{\}}%
}

\NewDocumentCommand\pkg{m}{\textsf{#1}}

\NewDocumentCommand\nan{}{\texttt{NaN}}
\NewDocumentCommand\enquote{m}{``#1''}

\let\tn=\cs

\RenewDocumentCommand\emph{m}{%
  \underline{\textsl{#1}}%
}

\newenvironment{l3regex-syntax}
  {\begin{itemize}\def\\{\char`\\}\def\makelabel##1{\hss\llap{\ttfamily##1}}}
  {\end{itemize}}

\usepackage{shortvrb}

\NewDocumentCommand \MyMakeShortVerb {} {%
  \MakeShortVerb \|%
  \MakeShortVerb \"%
}
\NewDocumentCommand \MyDeleteShortVerb {} {%
  \DeleteShortVerb \|%
  \DeleteShortVerb \"%
}

\AtBeginDocument{\MyMakeShortVerb}
\AtEndDocument{\MyDeleteShortVerb}
\AddToHook{env/demohigh/before}{\MyDeleteShortVerb}
\AddToHook{env/demohigh/after}{\MyMakeShortVerb}

\usepackage{hologo}
\providecommand*\LuaTeX{\hologo{LuaTeX}}
\providecommand*\pdfTeX{\hologo{pdfTeX}}
\providecommand*\XeTeX{\hologo{XeTeX}}

\usepackage{amsmath}

\usepackage{hyperref}
\hypersetup{
  colorlinks=true,
  urlcolor=blue3,
  linkcolor=blue3,
}

\usepackage{functional}
%\Functional{scoping=false,tracing=true}

\CodeHigh{lite}

\begin{document}

\chapter{Token Lists (\texttt{l3tl})}

\section{Scratch Variables of Token Lists}

\begin{variable}{\lTmpaTl,\lTmpbTl,\lTmpcTl,\lTmpiTl,\lTmpjTl,\lTmpkTl}
Scratch token lists for local assignment. These are never used by
the \verb!functional! package, and so are safe for use with any
function. However, they may be overwritten by other
code and so should only be used for short-term storage.
\end{variable}

\begin{variable}{\gTmpaTl,\gTmpbTl,\gTmpcTl,\gTmpiTl,\gTmpjTl,\gTmpkTl}
Scratch token lists for global assignment. These are never used by
the \verb!functional! package, and so are safe for use with any
function. However, they may be overwritten by other
code and so should only be used for short-term storage.
\end{variable}

\section{Public Functions for Token Lists}

\begin{function}{\TlNew}
\begin{syntax}
\cs{TlNew} \meta{tl~var}
\end{syntax}
Creates a new \meta{tl~var} or raises an error if the
name is already taken. The declaration is global. The
\meta{tl~var} is initially empty.
\end{function}

\begin{function}{\TlUse}
\begin{syntax}
\cs{TlUse} \meta{tl~var}
\end{syntax}
Recovers the content of a \meta{tl~var} and returns the value.
An error is raised if the variable
does not exist or if it is invalid. Note that it is possible to use
a \meta{tl~var} directly without an accessor function.
\end{function}

\begin{function}{\TlSet}
\begin{syntax}
\cs{TlSet} \meta{tl~var} \Arg{tokens}
\end{syntax}
Sets \meta{tl~var} to contain \meta{tokens},
removing any previous content from the variable. For example
\begin{demohigh}
\TlSet\lTmpiTl{\IntMathMult{4}{5}}
\TlUse\lTmpiTl
\end{demohigh}
\end{function}

\begin{function}{\TlClear}
\begin{syntax}
\cs{TlClear} \meta{tl~var}
\end{syntax}
Clears all entries from the \meta{tl~var}. For example
\begin{demohigh}
\TlSet\lTmpjTl{One}
\TlClear\lTmpjTl
\TlSet\lTmpjTl{Two}
\TlUse\lTmpjTl
\end{demohigh}
\end{function}

\begin{function}{\TlPutLeft}
\begin{syntax}
\cs{TlPutLeft} \meta{tl~var} \Arg{tokens}
\end{syntax}
Appends \meta{tokens} to the left side of the current content of
\meta{tl~var}. For example
\begin{demohigh}
\TlSet\lTmpkTl{Functional}
\TlPutLeft\lTmpkTl{Hello}
\TlUse\lTmpkTl
\end{demohigh}
\end{function}

\begin{function}{\TlPutRight}
\begin{syntax}
\cs{TlPutRight} \meta{tl~var} \Arg{tokens}
\end{syntax}
Appends \meta{tokens} to the right side of the current content of
\meta{tl~var}. For example
\begin{demohigh}
\TlSet\lTmpkTl{Functional}
\TlPutRight\lTmpkTl{World}
\TlUse\lTmpkTl
\end{demohigh}
\end{function}

\begin{function}{\TlIfEmpty,\TlIfEmptyTF}
\begin{syntax}
\cs{TlIfEmpty} \meta{tl~var}
\cs{TlIfEmptyTF} \meta{tl~var} \Arg{true code} \Arg{false code}
\end{syntax}
Tests if the \meta{token list variable} is entirely empty
(\emph{i.e.}~contains no tokens at all). For example
\begin{demohigh}
\TlSet\lTmpaTl{abc}
\TlIfEmptyTF\lTmpaTl{\Result{Empty}}{\Result{NonEmpty}}
\TlClear\lTmpaTl
\TlIfEmptyTF\lTmpaTl{\Result{Empty}}{\Result{NonEmpty}}
\end{demohigh}
\end{function}

\begin{function}{\TlIfEq,\TlIfEqTF}
\begin{syntax}
\cs{TlIfEq} \meta{tl~var_1} \meta{tl~var_2}
\cs{TlIfEqTF} \meta{tl~var_1} \meta{tl~var_2} \Arg{true code} \Arg{false code}
\end{syntax}
Compares the content of two \meta{token list variables} and
is logically \texttt{true} if the two contain the same list of
tokens (\emph{i.e.}~identical in both the list of characters they
contain and the category codes of those characters). For example
\begin{demohigh}
\TlSet\lTmpaTl{abc}
\TlSet\lTmpbTl{abc}
\TlSet\lTmpcTl{xyz}
\TlIfEqTF\lTmpaTl\lTmpbTl{\Result{Yes}}{\Result{No}}
\TlIfEqTF\lTmpaTl\lTmpcTl{\Result{Yes}}{\Result{No}}
\end{demohigh}
%See also \cs{StrIfEqTF} for a comparison that ignores category codes.
\end{function}

\end{document}
