% -*- coding: utf-8 -*-
% !TEX program = lualatex
\documentclass[oneside]{book}

% -*- coding: utf-8 -*-
% !TEX program = lualatex

\usepackage[a4paper,margin=2.5cm]{geometry}

\newcommand*{\myversion}{2022F}
\newcommand*{\mydate}{Version \myversion\ (\the\year-\mylpad\month-\mylpad\day)}
\newcommand*{\mylpad}[1]{\ifnum#1<10 0\the#1\else\the#1\fi}

\setlength{\parindent}{0pt}
\setlength{\parskip}{4pt plus 1pt minus 1pt}

\usepackage{codehigh}

\colorlet{highback}{blue9}
%\CodeHigh{lite}
\CodeHigh{language=latex/latex2,style/main=highback,style/code=highback}
\NewCodeHighEnv{code}{style/main=gray9,style/code=gray9}
\NewCodeHighEnv{demo}{style/main=gray9,style/code=gray9,demo}

\usepackage{enumitem}

\NewDocumentCommand\MySubScript{m}{$_{#1}$}

\ExplSyntaxOn
\NewDocumentCommand\PrintVarList{m}{
  \clist_set:Nn \l_tmpa_clist {#1}
  \clist_map_inline:Nn \l_tmpa_clist
    {
      \token_to_str:N ##1 ~
    }
}
\NewDocumentCommand\RelaceChacters{m}{
  \tl_set:Nn \lTmpaTl {#1}
  \regex_replace_once:nnN { \_ } { \c{MySubScript} } \lTmpaTl
}
\NewDocumentCommand\RelaceUnderScoreAll{m}{
  \tl_set:Nn \lTmpaTl {#1}
  \regex_replace_all:nnN { \_ } { \c{_} } \lTmpaTl
  \lTmpaTl
}
\ExplSyntaxOff

\NewDocumentEnvironment{variable}{om}{
  \vspace{5pt}
  \begin{minipage}{\linewidth}
  \hrule\vspace{4pt}\obeylines%
  \begingroup
  \ttfamily\bfseries\color{azure3}
  \PrintVarList{#2}
  \endgroup
  \par\vspace{4pt}\hrule
  \end{minipage}\par\nopagebreak\vspace{4pt}
}{%
  \vspace{5pt}%
}

\NewDocumentEnvironment{function}{om}{
  \vspace{5pt}%
}{\vspace{5pt}}

\NewDocumentEnvironment{syntax}{}{%
  \begin{minipage}{\linewidth}
  \hrule\vspace{4pt}\obeylines%
}{%
  \par\vspace{4pt}\hrule
  \end{minipage}\par\nopagebreak\vspace{4pt}
}

\NewDocumentEnvironment{texnote}{}{}{}

\NewDocumentCommand\cs{O{}m}{%
  \texttt{\bfseries\color{purple3}\cBackslashStr\RelaceUnderScoreAll{#2}}%
}
\NewDocumentCommand\meta{m}{%
  \RelaceChacters{#1}%
  \textsl{$\langle$\ignorespaces\lTmpaTl\unskip$\rangle$}%
}
\NewDocumentCommand\Arg{m}{%
  \RelaceChacters{#1}%
  \texttt{\{}\textsl{$\langle$\ignorespaces\lTmpaTl\unskip$\rangle$}\texttt{\}}%
}

\NewDocumentCommand\pkg{m}{\textsf{#1}}

\NewDocumentCommand\nan{}{\texttt{NaN}}
\NewDocumentCommand\enquote{m}{``#1''}

\let\tn=\cs

\RenewDocumentCommand\emph{m}{%
  \underline{\textsl{#1}}%
}

\newenvironment{l3regex-syntax}
  {\begin{itemize}\def\\{\char`\\}\def\makelabel##1{\hss\llap{\ttfamily##1}}}
  {\end{itemize}}

\usepackage{shortvrb}

\NewDocumentCommand \MyMakeShortVerb {} {%
  \MakeShortVerb \|%
  \MakeShortVerb \"%
}
\NewDocumentCommand \MyDeleteShortVerb {} {%
  \DeleteShortVerb \|%
  \DeleteShortVerb \"%
}

\AtBeginDocument{\MyMakeShortVerb}
\AtEndDocument{\MyDeleteShortVerb}
\AddToHook{env/demohigh/before}{\MyDeleteShortVerb}
\AddToHook{env/demohigh/after}{\MyMakeShortVerb}

\usepackage{hologo}
\providecommand*\LuaTeX{\hologo{LuaTeX}}
\providecommand*\pdfTeX{\hologo{pdfTeX}}
\providecommand*\XeTeX{\hologo{XeTeX}}

\usepackage{amsmath}

\usepackage{hyperref}
\hypersetup{
  colorlinks=true,
  urlcolor=blue3,
  linkcolor=blue3,
}

\usepackage{functional}
%\Functional{scoping=false,tracing=true}

\CodeHigh{lite}

\begin{document}

\chapter{Control Structures (\texttt{Bool})}

\section{Constant and Scratch Booleans}

\begin{variable}{\cTrueBool,\cFalseBool}
Constants that represent \verb|true| and \verb|false|, respectively. Used to
implement predicates. For example
\begin{demohigh}
\BoolVarIfTF \cTrueBool {\Return{True!}} {\Return{False!}}
\BoolVarIfTF \cFalseBool {\Return{True!}} {\Return{False!}}
\end{demohigh}
\end{variable}

\begin{variable}{\lTmpaBool,\lTmpbBool,\lTmpcBool,\lTmpiBool,\lTmpjBool,\lTmpkBool}
Scratch booleans for local assignment. These are never used by
the \verb!functional! package, and so are safe for use with any
function. However, they may be overwritten by other
code and so should only be used for short-term storage.
\end{variable}

\begin{variable}{\gTmpaBool,\gTmpbBool,\gTmpcBool,\gTmpiBool,\gTmpjBool,\gTmpkBool}
Scratch booleans for global assignment. These are never used by
the \verb!functional! package, and so are safe for use with any
function. However, they may be overwritten by other
code and so should only be used for short-term storage.
\end{variable}

\section{Creating and Setting Booleans}

\begin{function}{\BoolNew}
\begin{syntax}
\cs{BoolNew} \meta{boolean}
\end{syntax}
Creates a new \meta{boolean} or raises an error if the
name is already taken. The declaration is global. The
\meta{boolean} is initially \texttt{false}.
\end{function}

\begin{function}{\BoolConst}
\begin{syntax}
\cs{BoolConst} \meta{boolean} \Arg{boolexpr}
\end{syntax}
Creates a new constant \meta{boolean} or raises an error if the name
is already taken. The value of the \meta{boolean} is set globally to
the result of evaluating the \meta{boolexpr}.
For example
\begin{codehigh}
\BoolConst \cFooSomeBool {\IntCompare{3}>{2}}
\BoolVarLog \cFooSomeBool
\end{codehigh}
\end{function}

\begin{function}{\BoolSet}
\begin{syntax}
\cs{BoolSet} \meta{boolean} \Arg{boolexpr}
\end{syntax}
Evaluates the \meta{boolean expression} and sets the \meta{boolean} variable to
the logical truth of this evaluation.
For example
\begin{codehigh}
\BoolSet \lTmpaBool {\IntCompare{3}<{2}}
\BoolVarLog \lTmpaBool
\end{codehigh}
\end{function}

\begin{function}{\BoolSetTrue}
\begin{syntax}
\cs{BoolSetTrue} \meta{boolean}
\end{syntax}
Sets \meta{boolean} logically \texttt{true}.
\end{function}

\begin{function}{\BoolSetFalse}
\begin{syntax}
\cs{BoolSetFalse} \meta{boolean}
\end{syntax}
Sets \meta{boolean} logically \texttt{false}.
\end{function}

\begin{function}{\BoolSetEq}
\begin{syntax}
\cs{BoolSetEq} \meta{boolean_1} \meta{boolean_2}
\end{syntax}
Sets \meta{boolean_1} to the current value of \meta{boolean_2}.
For example
\begin{codehigh}
\BoolSetTrue \lTmpaBool
\BoolSetEq \lTmpbBool \lTmpaBool
\BoolVarLog \lTmpbBool
\end{codehigh}
\end{function}

\section{Viewing Booleans}

\begin{function}{\BoolLog}
\begin{syntax}
\cs{BoolLog} \Arg{boolean expression}
\end{syntax}
Writes the logical truth of the \meta{boolean expression} in the log file.
\end{function}

\begin{function}{\BoolVarLog}
\begin{syntax}
\cs{BoolVarLog} \meta{boolean}
\end{syntax}
Writes the logical truth of the \meta{boolean} in the log file.
\end{function}

\begin{function}{\BoolShow}
\begin{syntax}
\cs{BoolShow} \Arg{boolean expression}
\end{syntax}
Displays the logical truth of the \meta{boolean expression} on the terminal.
\end{function}

\begin{function}{\BoolVarShow}
\begin{syntax}
\cs{BoolVarShow} \meta{boolean}
\end{syntax}
Displays the logical truth of the \meta{boolean} on the terminal.
\end{function}

\section{Booleans and Conditionals}

\begin{function}{\BoolIfExist,\BoolIfExistT,\BoolIfExistF,\BoolIfExistTF}
\begin{syntax}
\cs{BoolIfExist} \meta{boolean}
\cs{BoolIfExistT} \meta{boolean} \Arg{true code}
\cs{BoolIfExistF} \meta{boolean} \Arg{false code}
\cs{BoolIfExistTF} \meta{boolean} \Arg{true code} \Arg{false code}
\end{syntax}
Tests whether the \meta{boolean} is currently defined.  This does not
check that the \meta{boolean} really is a boolean variable.
For example
\begin{demohigh}
\BoolIfExistTF \lTmpaBool {\Return{Yes}} {\Return{No}}
\BoolIfExistTF \lFooUndefinedBool {\Return{Yes}} {\Return{No}}
\end{demohigh}
\end{function}

\begin{function}{\BoolVarIf,\BoolVarIfT,\BoolVarIfF,\BoolVarIfTF}
\begin{syntax}
\cs{BoolVarIf} \meta{boolean}
\cs{BoolVarIfT} \meta{boolean} \Arg{true code}
\cs{BoolVarIfF} \meta{boolean} \Arg{false code}
\cs{BoolVarIfTF} \meta{boolean} \Arg{true code} \Arg{false code}
\end{syntax}
Tests the current truth of \meta{boolean}, and continues evaluation
based on this result. For example
\begin{demohigh}
\BoolSetTrue \lTmpaBool
\BoolVarIfTF \lTmpaBool {\Return{True!}} {\Return{False!}}
\BoolSetFalse \lTmpaBool
\BoolVarIfTF \lTmpaBool {\Return{True!}} {\Return{False!}}
\end{demohigh}
\end{function}

\begin{function}{\BoolVarNot,\BoolVarNotT,\BoolVarNotF,\BoolVarNotTF}
\begin{syntax}
\cs{BoolVarNot} \meta{boolean}
\cs{BoolVarNotT} \meta{boolean} \Arg{true code}
\cs{BoolVarNotF} \meta{boolean} \Arg{false code}
\cs{BoolVarNotTF} \meta{boolean} \Arg{true code} \Arg{false code}
\end{syntax}
Evaluates \meta{true code} if \meta{boolean} is \verb!false!,
and \meta{false code} if \meta{boolean} is \verb!true!.
For example
\begin{demohigh}
\BoolVarNotTF {\IntCompare{3}>{2}} {\Return{Yes}} {\Return{No}}
\end{demohigh}
\end{function}

\begin{function}{\BoolVarAnd,\BoolVarAndT,\BoolVarAndF,\BoolVarAndTF}
\begin{syntax}
\cs{BoolVarAnd} \meta{boolean_1} \meta{boolean_2}
\cs{BoolVarAndT} \meta{boolean_1} \meta{boolean_2} \Arg{true code}
\cs{BoolVarAndF} \meta{boolean_1} \meta{boolean_2} \Arg{false code}
\cs{BoolVarAndTF} \meta{boolean_1} \meta{boolean_2} \Arg{true code} \Arg{false code}
\end{syntax}
Implements the \enquote{And} operation between two booleans,
hence is \texttt{true} if both are \texttt{true}.
%Contrarily to the infix operator \verb|&&|,
The \meta{boolean_2} is only evaluated if it is needed to determine the result of
\cs{BoolVarAnd}.
For example
\begin{demohigh}
\BoolVarAndTF {\IntCompare{3}>{2}} {\IntCompare{3}>{4}} {\Return{Yes}} {\Return{No}}
\end{demohigh}
\end{function}

\begin{function}{\BoolVarOr,\BoolVarOrT,\BoolVarOrF,\BoolVarOrTF}
\begin{syntax}
\cs{BoolVarOr} \meta{boolean_1} \meta{boolean_2}
\cs{BoolVarOrT} \meta{boolean_1} \meta{boolean_2} \Arg{true code}
\cs{BoolVarOrF} \meta{boolean_1} \meta{boolean_2} \Arg{false code}
\cs{BoolVarOrTF} \meta{boolean_1} \meta{boolean_2} \Arg{true code} \Arg{false code}
\end{syntax}
Implements the \enquote{Or} operation between two booleans,
hence is \texttt{true} if either one is \texttt{true}.
%Contrarily to the infix operator \verb"||",
The \meta{boolean_2} is only evaluated if it is needed to determine the result of
\cs{BoolVarOr}.
For example
\begin{demohigh}
\BoolVarOrTF {\IntCompare{3}>{2}} {\IntCompare{3}>{4}} {\Return{Yes}} {\Return{No}}
\end{demohigh}
\end{function}

\begin{function}{\BoolVarXor,\BoolVarXorT,\BoolVarXorF,\BoolVarXorTF}
\begin{syntax}
\cs{BoolVarXor} \meta{boolean_1} \meta{boolean_2}
\cs{BoolVarXorT} \meta{boolean_1} \meta{boolean_2} \Arg{true code}
\cs{BoolVarXorF} \meta{boolean_1} \meta{boolean_2} \Arg{false code}
\cs{BoolVarXorTF} \meta{boolean_1} \meta{boolean_2} \Arg{true code} \Arg{false code}
\end{syntax}
Implements an \enquote{exclusive or} operation between two booleans.
For example
\begin{demohigh}
\BoolVarXorTF {\IntCompare{3}>{2}} {\IntCompare{3}>{4}} {\Return{Yes}} {\Return{No}}
\end{demohigh}
\end{function}

\end{document}
