% -*- coding: utf-8 -*-
% !TEX program = lualatex
\documentclass[oneside]{book}

% -*- coding: utf-8 -*-
% !TEX program = lualatex

\usepackage[a4paper,margin=2.5cm]{geometry}

\newcommand*{\myversion}{2022F}
\newcommand*{\mydate}{Version \myversion\ (\the\year-\mylpad\month-\mylpad\day)}
\newcommand*{\mylpad}[1]{\ifnum#1<10 0\the#1\else\the#1\fi}

\setlength{\parindent}{0pt}
\setlength{\parskip}{4pt plus 1pt minus 1pt}

\usepackage{codehigh}

\colorlet{highback}{blue9}
%\CodeHigh{lite}
\CodeHigh{language=latex/latex2,style/main=highback,style/code=highback}
\NewCodeHighEnv{code}{style/main=gray9,style/code=gray9}
\NewCodeHighEnv{demo}{style/main=gray9,style/code=gray9,demo}

\usepackage{enumitem}

\NewDocumentCommand\MySubScript{m}{$_{#1}$}

\ExplSyntaxOn
\NewDocumentCommand\PrintVarList{m}{
  \clist_set:Nn \l_tmpa_clist {#1}
  \clist_map_inline:Nn \l_tmpa_clist
    {
      \token_to_str:N ##1 ~
    }
}
\NewDocumentCommand\RelaceChacters{m}{
  \tl_set:Nn \lTmpaTl {#1}
  \regex_replace_once:nnN { \_ } { \c{MySubScript} } \lTmpaTl
}
\NewDocumentCommand\RelaceUnderScoreAll{m}{
  \tl_set:Nn \lTmpaTl {#1}
  \regex_replace_all:nnN { \_ } { \c{_} } \lTmpaTl
  \lTmpaTl
}
\ExplSyntaxOff

\NewDocumentEnvironment{variable}{om}{
  \vspace{5pt}
  \begin{minipage}{\linewidth}
  \hrule\vspace{4pt}\obeylines%
  \begingroup
  \ttfamily\bfseries\color{azure3}
  \PrintVarList{#2}
  \endgroup
  \par\vspace{4pt}\hrule
  \end{minipage}\par\nopagebreak\vspace{4pt}
}{%
  \vspace{5pt}%
}

\NewDocumentEnvironment{function}{om}{
  \vspace{5pt}%
}{\vspace{5pt}}

\NewDocumentEnvironment{syntax}{}{%
  \begin{minipage}{\linewidth}
  \hrule\vspace{4pt}\obeylines%
}{%
  \par\vspace{4pt}\hrule
  \end{minipage}\par\nopagebreak\vspace{4pt}
}

\NewDocumentEnvironment{texnote}{}{}{}

\NewDocumentCommand\cs{O{}m}{%
  \texttt{\bfseries\color{purple3}\cBackslashStr\RelaceUnderScoreAll{#2}}%
}
\NewDocumentCommand\meta{m}{%
  \RelaceChacters{#1}%
  \textsl{$\langle$\ignorespaces\lTmpaTl\unskip$\rangle$}%
}
\NewDocumentCommand\Arg{m}{%
  \RelaceChacters{#1}%
  \texttt{\{}\textsl{$\langle$\ignorespaces\lTmpaTl\unskip$\rangle$}\texttt{\}}%
}

\NewDocumentCommand\pkg{m}{\textsf{#1}}

\NewDocumentCommand\nan{}{\texttt{NaN}}
\NewDocumentCommand\enquote{m}{``#1''}

\let\tn=\cs

\RenewDocumentCommand\emph{m}{%
  \underline{\textsl{#1}}%
}

\newenvironment{l3regex-syntax}
  {\begin{itemize}\def\\{\char`\\}\def\makelabel##1{\hss\llap{\ttfamily##1}}}
  {\end{itemize}}

\usepackage{shortvrb}

\NewDocumentCommand \MyMakeShortVerb {} {%
  \MakeShortVerb \|%
  \MakeShortVerb \"%
}
\NewDocumentCommand \MyDeleteShortVerb {} {%
  \DeleteShortVerb \|%
  \DeleteShortVerb \"%
}

\AtBeginDocument{\MyMakeShortVerb}
\AtEndDocument{\MyDeleteShortVerb}
\AddToHook{env/demohigh/before}{\MyDeleteShortVerb}
\AddToHook{env/demohigh/after}{\MyMakeShortVerb}

\usepackage{hologo}
\providecommand*\LuaTeX{\hologo{LuaTeX}}
\providecommand*\pdfTeX{\hologo{pdfTeX}}
\providecommand*\XeTeX{\hologo{XeTeX}}

\usepackage{amsmath}

\usepackage{hyperref}
\hypersetup{
  colorlinks=true,
  urlcolor=blue3,
  linkcolor=blue3,
}

\usepackage{functional}
%\Functional{scoping=false,tracing=true}

\CodeHigh{lite}

\begin{document}

\chapter{Quarks (\texttt{Quark})}

%One special type of constants in \LaTeX3 is \enquote{quarks}.
%By convention all constants of type quark start out with \verb|\q|.

Quarks are control sequences (and in fact, token lists) that expand
to themselves and should therefore \emph{never} be executed directly
in the code. This would result in an endless loop!

Quarks can be used as error return values for functions that receive erroneous input.
For example, in the function \cs{PropGet} to retrieve a value stored
in some key of a property list, if the key does not exist then the return value
is the quark \cs{qNoValue}.
As mentioned above, such quarks are extremely fragile and it is imperative
when using such functions that code is carefully written to check for
pathological cases to avoid leakage of a quark into an uncontrolled
environment.

\section{Constant Quarks}

\begin{variable}{\qNoValue}
A canonical value for a missing value, when one is requested from
a data structure. This is therefore used as a \enquote{return} value
by functions such as \cs{PropGet} if there is no data to return.
\end{variable}

\section{Quark Conditionals}

\begin{function}{\QuarkVarIfNoValue,\QuarkVarIfNoValueTF}
\begin{syntax}
\cs{QuarkVarIfNoValue} \meta{token}
\cs{QuarkVarIfNoValueTF} \meta{token} \Arg{true code} \Arg{false code}
\end{syntax}
Tests if the \meta{token} is equal to \cs{qNoValue}.
\begin{demohigh}
\ClistGet \cEmptyClist \lTmpaTl
\QuarkVarIfNoValueTF \lTmpaTl {\Return{NoValue}} {\Return{SomeValue}}
\end{demohigh}
\begin{demohigh}
\SeqPop \cEmptySeq \lTmpaTl
\QuarkVarIfNoValueTF \lTmpaTl {\Return{NoValue}} {\Return{SomeValue}}
\end{demohigh}
\begin{demohigh}
\PropSetFromKeyval \lTmpaProp {key1=one,key2=two}
\PropGet \lTmpaProp {key3} \lTmpaTl
\QuarkVarIfNoValueTF \lTmpaTl {\Return{NoValue}} {\Return{SomeValue}}
\end{demohigh}
\end{function}

%\begin{function}{\QuarkIfNoValue,\QuarkIfNoValueTF}
%\begin{syntax}
%\cs{QuarkIfNoValue} \Arg{token list}
%\cs{QuarkIfNoValueTF} \Arg{token list} \Arg{true code} \Arg{false code}
%\end{syntax}
%Tests if the \meta{token list} contains only \cs{qNoValue}
%(distinct from \meta{token list} being empty or containing
%\cs{qNoValue} plus one or more other tokens).
%\end{function}

\end{document}
