% -*- coding: utf-8 -*-
% !TEX program = lualatex
\documentclass[oneside]{book}

% -*- coding: utf-8 -*-
% !TEX program = lualatex

\usepackage[a4paper,margin=2.5cm]{geometry}

\newcommand*{\myversion}{2022F}
\newcommand*{\mydate}{Version \myversion\ (\the\year-\mylpad\month-\mylpad\day)}
\newcommand*{\mylpad}[1]{\ifnum#1<10 0\the#1\else\the#1\fi}

\setlength{\parindent}{0pt}
\setlength{\parskip}{4pt plus 1pt minus 1pt}

\usepackage{codehigh}

\colorlet{highback}{blue9}
%\CodeHigh{lite}
\CodeHigh{language=latex/latex2,style/main=highback,style/code=highback}
\NewCodeHighEnv{code}{style/main=gray9,style/code=gray9}
\NewCodeHighEnv{demo}{style/main=gray9,style/code=gray9,demo}

\usepackage{enumitem}

\NewDocumentCommand\MySubScript{m}{$_{#1}$}

\ExplSyntaxOn
\NewDocumentCommand\PrintVarList{m}{
  \clist_set:Nn \l_tmpa_clist {#1}
  \clist_map_inline:Nn \l_tmpa_clist
    {
      \token_to_str:N ##1 ~
    }
}
\NewDocumentCommand\RelaceChacters{m}{
  \tl_set:Nn \lTmpaTl {#1}
  \regex_replace_once:nnN { \_ } { \c{MySubScript} } \lTmpaTl
}
\NewDocumentCommand\RelaceUnderScoreAll{m}{
  \tl_set:Nn \lTmpaTl {#1}
  \regex_replace_all:nnN { \_ } { \c{_} } \lTmpaTl
  \lTmpaTl
}
\ExplSyntaxOff

\NewDocumentEnvironment{variable}{om}{
  \vspace{5pt}
  \begin{minipage}{\linewidth}
  \hrule\vspace{4pt}\obeylines%
  \begingroup
  \ttfamily\bfseries\color{azure3}
  \PrintVarList{#2}
  \endgroup
  \par\vspace{4pt}\hrule
  \end{minipage}\par\nopagebreak\vspace{4pt}
}{%
  \vspace{5pt}%
}

\NewDocumentEnvironment{function}{om}{
  \vspace{5pt}%
}{\vspace{5pt}}

\NewDocumentEnvironment{syntax}{}{%
  \begin{minipage}{\linewidth}
  \hrule\vspace{4pt}\obeylines%
}{%
  \par\vspace{4pt}\hrule
  \end{minipage}\par\nopagebreak\vspace{4pt}
}

\NewDocumentEnvironment{texnote}{}{}{}

\NewDocumentCommand\cs{O{}m}{%
  \texttt{\bfseries\color{purple3}\cBackslashStr\RelaceUnderScoreAll{#2}}%
}
\NewDocumentCommand\meta{m}{%
  \RelaceChacters{#1}%
  \textsl{$\langle$\ignorespaces\lTmpaTl\unskip$\rangle$}%
}
\NewDocumentCommand\Arg{m}{%
  \RelaceChacters{#1}%
  \texttt{\{}\textsl{$\langle$\ignorespaces\lTmpaTl\unskip$\rangle$}\texttt{\}}%
}

\NewDocumentCommand\pkg{m}{\textsf{#1}}

\NewDocumentCommand\nan{}{\texttt{NaN}}
\NewDocumentCommand\enquote{m}{``#1''}

\let\tn=\cs

\RenewDocumentCommand\emph{m}{%
  \underline{\textsl{#1}}%
}

\newenvironment{l3regex-syntax}
  {\begin{itemize}\def\\{\char`\\}\def\makelabel##1{\hss\llap{\ttfamily##1}}}
  {\end{itemize}}

\usepackage{shortvrb}

\NewDocumentCommand \MyMakeShortVerb {} {%
  \MakeShortVerb \|%
  \MakeShortVerb \"%
}
\NewDocumentCommand \MyDeleteShortVerb {} {%
  \DeleteShortVerb \|%
  \DeleteShortVerb \"%
}

\AtBeginDocument{\MyMakeShortVerb}
\AtEndDocument{\MyDeleteShortVerb}
\AddToHook{env/demohigh/before}{\MyDeleteShortVerb}
\AddToHook{env/demohigh/after}{\MyMakeShortVerb}

\usepackage{hologo}
\providecommand*\LuaTeX{\hologo{LuaTeX}}
\providecommand*\pdfTeX{\hologo{pdfTeX}}
\providecommand*\XeTeX{\hologo{XeTeX}}

\usepackage{amsmath}

\usepackage{hyperref}
\hypersetup{
  colorlinks=true,
  urlcolor=blue3,
  linkcolor=blue3,
}

\usepackage{functional}
%\Functional{scoping=false,tracing=true}

\CodeHigh{lite}

\begin{document}

\chapter{Floating Point Numbers (\texttt{Fp})}

\section{Constant and Scratch Floating Points}

\begin{variable}{\cZeroFp, \cMinusZeroFp}
Zero, with either sign.
\end{variable}

\begin{variable}{\cOneFp}
One as an \texttt{fp}: useful for comparisons in some places.
\end{variable}

\begin{variable}{\cInfFp,\cMinusInfFp}
Infinity, with either sign. These can be input directly in a
floating point expression as \texttt{inf} and \texttt{-inf}.
\end{variable}

\begin{variable}{\cEFp}
The value of the base of the natural logarithm, $\mathrm{e} = \exp(1)$.
\end{variable}

\begin{variable}{\cPiFp}
The value of $\pi$.  This can be input directly in a floating point
expression as \texttt{pi}.
\end{variable}

\begin{variable}{\cOneDegreeFp}
The value of $1^{\circ}$ in radians. Multiply an angle given in
degrees by this value to obtain a result in radians.  Note that
trigonometric functions expecting an argument in radians or in
degrees are both available.  Within floating point expressions, this
can be accessed as \texttt{deg}.
\end{variable}

\begin{variable}{\lTmpaFp,\lTmpbFp,\lTmpcFp,\lTmpiFp,\lTmpjFp,\lTmpkFp}
Scratch floating point numbers for local assignment. These are never used by
the \verb!functional! package, and so are safe for use with any
function. However, they may be overwritten by other
code and so should only be used for short-term storage.
\end{variable}

\begin{variable}{\gTmpaFp,\gTmpbFp,\gTmpcFp,\gTmpiFp,\gTmpjFp,\gTmpkFp}
Scratch floating point numbers for global assignment. These are never used by
the \verb!functional! package, and so are safe for use with any
function. However, they may be overwritten by other
code and so should only be used for short-term storage.
\end{variable}

\section{Floating Point Expressions}

\begin{function}{\FpEval}
\begin{syntax}
\cs{FpEval} \Arg{floating point expression}
\end{syntax}
Evaluates the \meta{floating point expression} and returns the
result as a decimal number with no
exponent.  Leading or trailing zeros may be inserted to compensate
for the exponent.  Non-significant trailing zeros are trimmed, and
integers are expressed without a decimal separator.  The values
$\pm\infty$ and \nan{} trigger an \enquote{invalid operation}
exception.
For a tuple, each item is converted using \cs{FpEval} and they are combined as
\verb|(|\meta{fp_1}\verb*|, |\meta{fp_2}\verb*|, |\ldots{}\meta{fp_n}\verb|)|
if $n>1$ and \verb|(|\meta{fp_1}\verb|,)| or \verb|()| for fewer items.
%This function is identical to \cs{fp_to_decimal:n}.
For example
\begin{demohigh}
\FpEval {(1.2+3.4)*(5.6-7.8)/9}
\end{demohigh}
\end{function}

\begin{function}{\FpMathAdd}
\begin{syntax}
\cs{FpMathAdd} \Arg{fpexpr_1} \Arg{fpexpr_2}
\end{syntax}
Adds \Arg{fpexpr_1} and \Arg{fpexpr_2},
and returns the result. For example
\begin{demohigh}
\FpMathAdd {2.8} {3.7}
\FpMathAdd {3.8-1} {2.7+1}
\end{demohigh}
\end{function}

\begin{function}{\FpMathSub}
\begin{syntax}
\cs{FpMathSub} \Arg{fpexpr_1} \Arg{fpexpr_2}
\end{syntax}
Subtracts \Arg{fpexpr_2} from \Arg{fpexpr_1},
and returns the result. For example
\begin{demohigh}
\FpMathSub {2.8} {3.7}
\FpMathSub {3.8-1} {2.7+1}
\end{demohigh}
\end{function}

\begin{function}{\FpMathMult}
\begin{syntax}
\cs{FpMathMult} \Arg{fpexpr_1} \Arg{fpexpr_2}
\end{syntax}
Multiplies \Arg{fpexpr_1} by \Arg{fpexpr_2},
and returns the result. For example
\begin{demohigh}
\FpMathMult {2.8} {3.7}
\FpMathMult {3.8-1} {2.7+1}
\end{demohigh}
\end{function}

\begin{function}{\FpMathDiv}
\begin{syntax}
\cs{FpMathDiv} \Arg{fpexpr_1} \Arg{fpexpr_2}
\end{syntax}
Divides \Arg{fpexpr_1} by \Arg{fpexpr_2},
and returns the result. For example
\begin{demohigh}
\FpMathDiv {2.8} {3.7}
\FpMathDiv {3.8-1} {2.7+1}
\end{demohigh}
\end{function}

\begin{function}{\FpMathSign}
\begin{syntax}
\cs{FpMathSign} \Arg{fpexpr}
\end{syntax}
Evaluates the \meta{fpexpr} and returns the value
using \cs{FpEval}\verb|{sign(|\meta{result}\verb|)}|: $+1$ for positive
numbers and for $+\infty$, $-1$ for negative numbers and for
$-\infty$, $\pm 0$ for $\pm 0$.  If the operand is a tuple or is
\nan{}, then \enquote{invalid operation} occurs and the result
is $0$. For example
\begin{demohigh}
\FpMathSign {3.5}
\FpMathSign {-2.7}
\end{demohigh}
\end{function}

\begin{function}{\FpMathAbs}
\begin{syntax}
\cs{FpMathAbs} \Arg{floating point expression}
\end{syntax}
Evaluates the \meta{floating point expression} as described for
\cs{FpEval} and returns the absolute value.
If the argument is $\pm\infty$, \nan{} or a tuple,
\enquote{invalid operation} occurs.  Within floating point
expressions, \verb|abs()| can be used; it accepts $\pm\infty$ and \nan{}
as arguments.
\end{function}

\begin{function}{\FpMathMax,\FpMathMin}
\begin{syntax}
\cs{FpMathMax} \Arg{fp expression_1} \Arg{fp expression_2}
\cs{FpMathMin} \Arg{fp expression_1} \Arg{fp expression_2}
\end{syntax}
Evaluates the \meta{floating point expressions} as described for \cs{FpEval}
and returns the resulting larger (\texttt{max}) or smaller (\texttt{min}) value.
If the argument is a tuple, \enquote{invalid operation} occurs,
but no other case raises exceptions. Within floating point expressions,
\verb|max()| and \verb|min()| can be used.
\end{function}

\section{Creating and Using Floating Points}

\begin{function}{\FpNew}
\begin{syntax}
\cs{FpNew} \meta{fp var}
\end{syntax}
Creates a new \meta{fp var} or raises an error if the name is
already taken. The declaration is global. The \meta{fp~var} is
initially $+0$.
\end{function}

\begin{function}{\FpConst}
\begin{syntax}
\cs{FpConst} \meta{fp var} \Arg{floating point expression}
\end{syntax}
Creates a new constant \meta{fp var} or raises an error if the name
is already taken. The \meta{fp var} is set globally equal to
the result of evaluating the \meta{floating point expression}.
For example
\begin{demohigh}
\FpConst \cMyPiFp {3.1415926}
\FpUse \cMyPiFp
\end{demohigh}
\end{function}

\begin{function}{\FpUse}
\begin{syntax}
\cs{FpUse} \meta{fp var}
\end{syntax}
Recovers the value of the \meta{fp var} and returns the value as a
decimal number with no exponent.
%Leading or trailing zeros may be inserted to compensate for the
%exponent.  Non-significant trailing zeros are trimmed.  Integers are
%expressed without a decimal separator.  The values $\pm\infty$
%and \nan{} trigger an \enquote{invalid operation} exception.
%For a tuple, each item is converted using \cs{fp_to_decimal:n} and they are combined as
%|(|\meta{fp_1}\verb*|, |\meta{fp_2}\verb*|, |\ldots{}\meta{fp_n}|)|
%if $n>1$ and |(|\meta{fp_1}|,)| or |()| for fewer items.
%This function is identical to \cs{fp_to_decimal:N}.
\end{function}

\section{Viewing Floating Points}

\begin{function}{\FpLog}
\begin{syntax}
\cs{FpLog} \Arg{floating point expression}
\end{syntax}
Evaluates the \meta{floating point expression} and writes the
result in the log file.
\end{function}

\begin{function}{\FpVarLog}
\begin{syntax}
\cs{FpVarLog} \meta{fp var}
\end{syntax}
Writes the value of \meta{fp var} in the log file.
\end{function}

\begin{function}{\FpShow}
\begin{syntax}
\cs{FpShow} \Arg{floating point expression}
\end{syntax}
Evaluates the \meta{floating point expression} and displays the
result in the terminal.
\end{function}

\begin{function}{\FpVarShow}
\begin{syntax}
\cs{FpVarShow} \meta{fp var}
\end{syntax}
Displays the value of \meta{fp var} in the terminal.
\end{function}

\section{Setting Floating Point Variables}

\begin{function}{\FpSet}
\begin{syntax}
\cs{FpSet} \meta{fp var} \Arg{floating point expression}
\end{syntax}
Sets \meta{fp var} equal to the result of computing the
\meta{floating point expression}. For example
\begin{demohigh}
\FpSet \lTmpaFp {4/7}
\FpUse \lTmpaFp
\end{demohigh}
\end{function}

\begin{function}{\FpSetEq}
\begin{syntax}
\cs{FpSetEq} \meta{fp var_1} \meta{fp var_2}
\end{syntax}
Sets the floating point variable \meta{fp var_1} equal to the current
value of \meta{fp var_2}.
\end{function}

\begin{function}{\FpZero}
\begin{syntax}
\cs{FpZero} \meta{fp var}
\end{syntax}
Sets the \meta{fp var} to $+0$. For example
\begin{demohigh}
\FpSet \lTmpaFp {5.3}
\FpZero \lTmpaFp
\FpUse \lTmpaFp
\end{demohigh}
\end{function}

\begin{function}{\FpZeroNew}
\begin{syntax}
\cs{FpZeroNew} \meta{fp var}
\end{syntax}
Ensures that the \meta{fp var} exists globally
by applying \cs{FpNew} if necessary, then applies
\cs{FpZero} to leave the \meta{fp var} set to $+0$.
\end{function}

\begin{function}{\FpAdd}
\begin{syntax}
\cs{FpAdd} \meta{fp var} \Arg{floating point expression}
\end{syntax}
Adds the result of computing the \meta{floating point expression} to
the \meta{fp var}.
This also applies if \meta{fp var} and \meta{floating point
expression} evaluate to tuples of the same size. For example
\begin{demohigh}
\FpSet \lTmpaFp {5.3}
\FpAdd \lTmpaFp {2.11}
\FpUse \lTmpaFp
\end{demohigh}
\end{function}

\begin{function}{\FpSub}
\begin{syntax}
\cs{FpSub} \meta{fp var} \Arg{floating point expression}
\end{syntax}
Subtracts the result of computing the \meta{floating point
expression} from the \meta{fp var}.
This also applies if \meta{fp var} and \meta{floating point
expression} evaluate to tuples of the same size. For example
\begin{demohigh}
\FpSet \lTmpaFp {5.3}
\FpSub \lTmpaFp {2.11}
\FpUse \lTmpaFp
\end{demohigh}
\end{function}

\section{Floating Point Step Functions}

%\begin{function}{\FpStepFunction}
%\begin{syntax}
%\cs{FpStepFunction} \Arg{initial value} \Arg{step} \Arg{final value} \meta{function}
%\end{syntax}
%This function first evaluates the \meta{initial value}, \meta{step}
%and \meta{final value}, each of which should be a floating point
%expression evaluating to a floating point number, not a tuple.
%The \meta{function} is then placed in front of each \meta{value}
%from the \meta{initial value} to the \meta{final value} in turn
%(using \meta{step} between each \meta{value}).  The \meta{step} must
%be non-zero.  If the \meta{step} is positive, the loop stops when
%the \meta{value} becomes larger than the \meta{final value}.  If the
%\meta{step} is negative, the loop stops when the \meta{value}
%becomes smaller than the \meta{final value}. The \meta{function}
%should absorb one numerical argument. For example
%\begin{verbatim}
%\cs_set:Npn \my_func:n #1 { [I saw #1] \quad }
%\fp_step_function:nnnN { 1.0 } { 0.1 } { 1.5 } \my_func:n
%\end{verbatim}
%would print
%\begin{quote}
%[I saw 1.0] \quad
%[I saw 1.1] \quad
%[I saw 1.2] \quad
%[I saw 1.3] \quad
%[I saw 1.4] \quad
%[I saw 1.5] \quad
%\end{quote}
%\begin{texnote}
%Due to rounding, it may happen that adding the \meta{step} to the
%\meta{value} does not change the \meta{value}; such cases give an
%error, as they would otherwise lead to an infinite loop.
%\end{texnote}
%\end{function}

\begin{function}{\FpStepInline}
\begin{syntax}
\cs{FpStepInline} \Arg{initial value} \Arg{step} \Arg{final value} \Arg{code}
\end{syntax}
This function first evaluates the \meta{initial value}, \meta{step}
and \meta{final value}, all of which should be floating point
expressions evaluating to a floating point number, not a tuple.
Then for each \meta{value} from the \meta{initial value} to the
\meta{final value} in turn (using \meta{step} between each
\meta{value}), the \meta{code} is inserted into the input stream
with \verb|#1| replaced by the current \meta{value}. Thus the
\meta{code} should define a function of one argument (\verb|#1|).
\begin{demohigh}
\IgnoreSpacesOn
\TlClear \lTmpaTl
\FpStepInline {1} {0.1} {1.5} {
  \TlPutRight \lTmpaTl {[#1]}
}
\TlUse \lTmpaTl
\IgnoreSpacesOff
\end{demohigh}
\end{function}

\begin{function}{\FpStepVariable}
\begin{syntax}
\cs{FpStepVariable} \Arg{initial value} \Arg{step} \Arg{final value} \meta{tl var} \Arg{code}
\end{syntax}
This function first evaluates the \meta{initial value}, \meta{step}
and \meta{final value}, all of which should be floating point
expressions evaluating to a floating point number, not a tuple.
Then for each \meta{value} from the \meta{initial value} to the
\meta{final value} in turn (using \meta{step} between each
\meta{value}), the \meta{code} is inserted into the input stream,
with the \meta{tl var} defined as the current \meta{value}.  Thus
the \meta{code} should make use of the \meta{tl var}.
%For example
%\begin{demohigh}
%\IgnoreSpacesOn
%\TlClear\lTmpaTl
%\FpStepVariable{1}{0.1}{1.5}\lTmpiTl{
%  \TlPutRight\lTmpaTl{\Value\lTmpiTl}
%  \TlPutRight\lTmpaTl{~}
%}
%\Return{\Value\lTmpaTl}
%\IgnoreSpacesOff
%\end{demohigh}
\end{function}

\section{Float Point Conditionals}

\begin{function}{\FpIfExist,\FpIfExistT,\FpIfExistF,\FpIfExistTF}
\begin{syntax}
\cs{FpIfExist} \meta{fp var}
\cs{FpIfExistT} \meta{fp var} \Arg{true code}
\cs{FpIfExistF} \meta{fp var} \Arg{false code}
\cs{FpIfExistTF} \meta{fp var} \Arg{true code} \Arg{false code}
\end{syntax}
Tests whether the \meta{fp var} is currently defined.  This does not
check that the \meta{fp var} really is a floating point variable.
For example
\begin{demohigh}
\FpIfExistTF \lTmpaFp {\Return{Yes}} {\Return{No}}
\FpIfExistTF \lMyUndefinedFp {\Return{Yes}} {\Return{No}}
\end{demohigh}
\end{function}

\begin{function}{\FpCompare,\FpCompareT,\FpCompareF,\FpCompareTF}
\begin{syntax}
\cs{FpCompare} \Arg{fpexpr_1} \meta{relation} \Arg{fpexpr_2}
\cs{FpCompareT} \Arg{fpexpr_1} \meta{relation} \Arg{fpexpr_2} \Arg{true code}
\cs{FpCompareF} \Arg{fpexpr_1} \meta{relation} \Arg{fpexpr_2} \Arg{false code}
\cs{FpCompareTF} \Arg{fpexpr_1} \meta{relation} \Arg{fpexpr_2} \Arg{true code} \Arg{false code}
\end{syntax}
Compares the \meta{fpexpr_1} and the \meta{fpexpr_2}, and returns
\texttt{true} if the \meta{relation} is obeyed. For example
\begin{demohigh}
\FpCompareTF {1} > {0.9999} {\Return{Greater}} {\Return{Less}}
\FpCompareTF {1} > {1.0001} {\Return{Greater}} {\Return{Less}}
\end{demohigh}
Two floating points
$x$ and $y$ may obey four mutually exclusive relations:
$x<y$, $x=y$, $x>y$, or $x?y$ (\enquote{not ordered}).  The last
case occurs exactly if one or both operands is \nan{} or is a tuple,
unless they are equal tuples.  Note that a \nan{} is distinct from
any value, even another \nan{}, hence $x=x$ is not true for
a \nan{}.  To test if a value is \nan{}, compare it to an arbitrary
number with the \enquote{not ordered} relation.\par
%\begin{demohigh}
%\FpCompareTF{0/0}?{0}{\Return{Is~a~Nan}}{\Return{Isn't~a~NaN}}
%\end{demohigh}
Tuples are equal if they have the same number of items and items
compare equal (in particular there must be no \nan{}).
At present any other comparison with tuples yields \verb|?| (not ordered).
This is experimental.
\end{function}

\end{document}
