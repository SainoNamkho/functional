% -*- coding: utf-8 -*-
% !TEX program = lualatex
\documentclass[oneside]{book}

% -*- coding: utf-8 -*-
% !TEX program = lualatex

\usepackage[a4paper,margin=2.5cm]{geometry}

\newcommand*{\myversion}{2022F}
\newcommand*{\mydate}{Version \myversion\ (\the\year-\mylpad\month-\mylpad\day)}
\newcommand*{\mylpad}[1]{\ifnum#1<10 0\the#1\else\the#1\fi}

\setlength{\parindent}{0pt}
\setlength{\parskip}{4pt plus 1pt minus 1pt}

\usepackage{codehigh}

\colorlet{highback}{blue9}
%\CodeHigh{lite}
\CodeHigh{language=latex/latex2,style/main=highback,style/code=highback}
\NewCodeHighEnv{code}{style/main=gray9,style/code=gray9}
\NewCodeHighEnv{demo}{style/main=gray9,style/code=gray9,demo}

\usepackage{enumitem}

\NewDocumentCommand\MySubScript{m}{$_{#1}$}

\ExplSyntaxOn
\NewDocumentCommand\PrintVarList{m}{
  \clist_set:Nn \l_tmpa_clist {#1}
  \clist_map_inline:Nn \l_tmpa_clist
    {
      \token_to_str:N ##1 ~
    }
}
\NewDocumentCommand\RelaceChacters{m}{
  \tl_set:Nn \lTmpaTl {#1}
  \regex_replace_once:nnN { \_ } { \c{MySubScript} } \lTmpaTl
}
\NewDocumentCommand\RelaceUnderScoreAll{m}{
  \tl_set:Nn \lTmpaTl {#1}
  \regex_replace_all:nnN { \_ } { \c{_} } \lTmpaTl
  \lTmpaTl
}
\ExplSyntaxOff

\NewDocumentEnvironment{variable}{om}{
  \vspace{5pt}
  \begin{minipage}{\linewidth}
  \hrule\vspace{4pt}\obeylines%
  \begingroup
  \ttfamily\bfseries\color{azure3}
  \PrintVarList{#2}
  \endgroup
  \par\vspace{4pt}\hrule
  \end{minipage}\par\nopagebreak\vspace{4pt}
}{%
  \vspace{5pt}%
}

\NewDocumentEnvironment{function}{om}{
  \vspace{5pt}%
}{\vspace{5pt}}

\NewDocumentEnvironment{syntax}{}{%
  \begin{minipage}{\linewidth}
  \hrule\vspace{4pt}\obeylines%
}{%
  \par\vspace{4pt}\hrule
  \end{minipage}\par\nopagebreak\vspace{4pt}
}

\NewDocumentEnvironment{texnote}{}{}{}

\NewDocumentCommand\cs{O{}m}{%
  \texttt{\bfseries\color{purple3}\cBackslashStr\RelaceUnderScoreAll{#2}}%
}
\NewDocumentCommand\meta{m}{%
  \RelaceChacters{#1}%
  \textsl{$\langle$\ignorespaces\lTmpaTl\unskip$\rangle$}%
}
\NewDocumentCommand\Arg{m}{%
  \RelaceChacters{#1}%
  \texttt{\{}\textsl{$\langle$\ignorespaces\lTmpaTl\unskip$\rangle$}\texttt{\}}%
}

\NewDocumentCommand\pkg{m}{\textsf{#1}}

\NewDocumentCommand\nan{}{\texttt{NaN}}
\NewDocumentCommand\enquote{m}{``#1''}

\let\tn=\cs

\RenewDocumentCommand\emph{m}{%
  \underline{\textsl{#1}}%
}

\newenvironment{l3regex-syntax}
  {\begin{itemize}\def\\{\char`\\}\def\makelabel##1{\hss\llap{\ttfamily##1}}}
  {\end{itemize}}

\usepackage{shortvrb}

\NewDocumentCommand \MyMakeShortVerb {} {%
  \MakeShortVerb \|%
  \MakeShortVerb \"%
}
\NewDocumentCommand \MyDeleteShortVerb {} {%
  \DeleteShortVerb \|%
  \DeleteShortVerb \"%
}

\AtBeginDocument{\MyMakeShortVerb}
\AtEndDocument{\MyDeleteShortVerb}
\AddToHook{env/demohigh/before}{\MyDeleteShortVerb}
\AddToHook{env/demohigh/after}{\MyMakeShortVerb}

\usepackage{hologo}
\providecommand*\LuaTeX{\hologo{LuaTeX}}
\providecommand*\pdfTeX{\hologo{pdfTeX}}
\providecommand*\XeTeX{\hologo{XeTeX}}

\usepackage{amsmath}

\usepackage{hyperref}
\hypersetup{
  colorlinks=true,
  urlcolor=blue3,
  linkcolor=blue3,
}

\usepackage{functional}
%\Functional{scoping=false,tracing=true}

\CodeHigh{lite}

\begin{document}

\chapter{Overview of Features}

\section{Evaluation from Inside to Outside}

We will compare our first example with a similar \verb!Lua! example:

\begin{code}
-- lua code --
function MathSquare (arg)
  local lTmpaInt = arg * arg
  return lTmpaInt
end
print(MathSquare(5))
print(MathSquare(MathSquare(5)))
\end{code}

\begin{codehigh}
%% function code
\ExplSyntaxOn
\PrgNewFunction \MathSquare { m } {
  \IntSet \lTmpaInt { \IntEval { #1 * #1 } }
  \Result { \Value \lTmpaInt }
}
\ExplSyntaxOff
\MathSquare{5}
\MathSquare{\MathSquare{5}}
\end{codehigh}

%\ExplSyntaxOn
%\PrgNewFunction \MathSquare { m }{
%  \IntSet \lTmpaInt { \IntEval { #1 * #1 } }
%  \Result { \Value \lTmpaInt }
%}
%\ExplSyntaxOff
%\MathSquare{5}
%\MathSquare{\MathSquare{5}}

%\ExplSyntaxOn
%\PrgNewFunction \MathCubic { m }
%  {
%    \IntSet \lTmpaInt { \IntEval { #1 * #1 * #1 } }
%    \Result { \Value \lTmpaInt }
%  }
%\ExplSyntaxOff
%\MathCubic{2}
%\MathCubic{\MathCubic{2}}

Both examples calculate first the square of $5$ and produce $25$,
then calculate the square of $25$ and produce $625$.
In contrast to \verb!expl3!, this \verb!functional! package
does evaluation of functions from inside to outside,
which means composition of functions works like othe programming languages
such as \verb!Lua! or \verb!JavsScript!.

You can define new functions with \cs{PrgNewFunction} command.
To make composition of functions work as expected,
every function \emph{must not} insert directly any token to the input stream.
Instead, a function \emph{must} pass the result (if any) to \verb!functional! package
with \cs{Result} command. And \verb!functional! package is responsible for
inserting result tokens to the input stream at the appropriate time.

To remove space tokens inside function code in defining functions,
you'd better put function definitions inside \verb!\ExplSyntaxOn! and
\verb!\ExplSyntaxOff! block. Within this block, \verb!~! is used to input a space.

At the end of this section,
we will compare our factorial example with a similar \verb!Lua! example:

\begin{code}
-- lua code --
function Factorial (n)
  if n == 0 then
    return 1
  else
    return n * Factorial(n-1)
  end
end
print(Factorial(4))
\end{code}

\begin{codehigh}
\ExplSyntaxOn
\PrgNewFunction \Factorial { m } {
  \IntCompareTF {#1} = {0} {
    \Result {1}
  }{
    \Result { \IntMathMult {#1} { \Factorial { \IntMathSub{#1}{1} } } }
  }
}
\ExplSyntaxOff
\Factorial{4}
\end{codehigh}

%\ExplSyntaxOn
%\PrgNewFunction \Factorial { m } {
%  \IntCompareTF {#1} = {0} {
%    \Result {1}
%  }{
%    \Result { \IntMathMult {#1} { \Factorial { \IntMathSub{#1}{1} } } }
%  }
%}
%\ExplSyntaxOff
%\Factorial{0}
%\Factorial{4}

\section{Group Scoping of Functions}

In \verb!Lua! language, a function or a condition expression makes a block,
and the values of local variables will be reset after a block.
For example

\begin{code}
-- lua code --
local a = 1
print(a)          ---- 1
function SomeFun()
  local a = 2
  print(a)        ---- 2
  if 1 > 0 then
    local a = 3
    print(a)      ---- 3
  end
  print(a)        ---- 2
end
SomeFun()
print(a)          ---- 1
\end{code}

In \verb!functional! package, a condition expression is in fact a function,
and you can make every function become a group by setting
\verb!\Functional{scoping=true}!. For example

\begin{codehigh}
\Functional{scoping=true}
\ExplSyntaxOn
\IntSet \lTmpaInt {1}
\IntLogVar \lTmpaInt          % ---- 1
\PrgNewFunction \SomeFun { } {
  \IntSet \lTmpaInt {2}
  \IntLogVar \lTmpaInt        % ---- 2
  \IntCompareTF {1} > {0} {
    \IntSet \lTmpaInt {3}
    \IntLogVar \lTmpaInt      % ---- 3
  }{ }
  \IntLogVar \lTmpaInt        % ---- 2
}
\SomeFun
\IntLogVar \lTmpaInt          % ---- 1
\ExplSyntaxOff
\end{codehigh}

Same as \verb!expl3!, the names of local variables \emph{must} start with \verb!l!,
while names of global variables \emph{must} start with \verb!g!.
The difference is that \verb!functional! package provides only one function for setting
both local and global varianbles of the same type,
by checking leading letters of their names. So for integer variables, you can write
\verb!\IntSet\lTmpaInt{1}! and \verb!\IntSet\gTmpbInt{2}!.

The previous example will produce different result
if we change variable from \verb!\lTmpaInt! to \verb!\gTmpaInt!.

\begin{codehigh}
\Functional{scoping=true}
\IntSet \gTmpaInt {1}
\IntLogVar \gTmpaInt          % ---- 1
\PrgNewFunction \SomeFun { } {
  \IntSet \gTmpaInt {2}
  \IntLogVar \gTmpaInt        % ---- 2
  \IntCompareTF {1} > {0} {
    \IntSet \gTmpaInt {3}
    \IntLogVar \gTmpaInt      % ---- 3
  }{ }
  \IntLogVar \gTmpaInt        % ---- 3
}
\SomeFun
\IntLogVar \gTmpaInt          % ---- 3
\end{codehigh}

As you can see, the values of global variables will never be reset after a group.

\section{Tracing Evaluation of Functions}

Since every function in \verb!functional! package will pass its return value to
the package, it is quite easy to debug your code.
You can turn on the tracing by setting \verb!\Functional{tracing=true}!.
For example, the tracing log of the first example in this chapter will be the following:

% FIXME: spaces at the first line will be removed
%\begin{codehigh}[]
%    [I] \MathSquare{5}
%            [I] \IntEval{5*5}
%                    [I] \Expand{\int_eval:n {5*5}}
%                    [O] 25
%                [I] \Result{25}
%                [O] 25
%            [O] 25
%        [I] \IntSet\lTmpaInt {25}
%        [O]
%            [I] \Value\lTmpaInt
%            [O] 25
%        [I] \Result{25}
%        [O] 25
%    [O] 25
%\end{codehigh}
\begin{codehigh}[]
[I] \MathSquare{5}
        [I] \IntEval{5*5}
                [I] \Expand{\int_eval:n {5*5}}
                [O] 25
            [I] \Result{25}
            [O] 25
        [O] 25
    [I] \IntSet\lTmpaInt {25}
    [O]
        [I] \Value\lTmpaInt
        [O] 25
    [I] \Result{25}
    [O] 25
[O] 25
[I] \MathSquare{25}
        [I] \IntEval{25*25}
                [I] \Expand{\int_eval:n {25*25}}
                [O] 625
            [I] \Result{625}
            [O] 625
        [O] 625
    [I] \IntSet\lTmpaInt {625}
    [O]
        [I] \Value\lTmpaInt
        [O] 625
    [I] \Result{625}
    [O] 625
[O] 625
\end{codehigh}

\section{Definitions of Functions}

Within \verb!expl3!, there are eight commands for defining new functions,
which is good for power users.

\begin{code}[language=latex/latex3]
\cs_new:Npn
\cs_new_nopar:Npn
\cs_new_protected:Npn
\cs_new_protected_nopar:Npn
\cs_new:Nn
\cs_new_nopar:Nn
\cs_new_protected:Nn
\cs_new_protected_nopar:Nn
\end{code}

Within \verb!functional! package, there is only one command (\cs{PrgNewFunction})
for defining new functions, which is good for normal users.
The created functions are always protected and accept \verb!\par! in their arguments.

Since \verb!functional! package gets the results of functions by evaluation
(including expansion and execution by \TeX), it is natural to protect all functions.

\section{Variants of Arguments}

Within \verb!expl3!, there are several expansion variants for arguments,
and many expansion functions for expanding them, which are necessary for power users.

\begin{code}[language=latex/latex3]
\module_foo:c
\module_bar:e
\module_bar:x
\module_bar:f
\module_bar:o
\module_bar:V
\module_bar:v
\end{code}

\begin{code}[language=latex/latex3]
\exp_args:Nc
\exp_args:Ne
\exp_args:Nx
\exp_args:Nf
\exp_args:No
\exp_args:NV
\exp_args:Nv
\end{code}

Within \verb!functional! package, there are only three variants
(\verb!c!, \verb!e!, \verb!V!) are provided, and these variants are defined
as functions (\cs{Name}, \cs{Expand}, \cs{Value}, respetively),
which are easier to use for normal users.

\begin{demohigh}
\newcommand\test{uvw}
\Name{test}
\end{demohigh}

\begin{demohigh}
\newcommand\test{uvw}
\Expand{111\test222}
\end{demohigh}

\begin{demohigh}
\IntSet\lTmpaInt{123}
\Value\lTmpaInt
\end{demohigh}

The most interesting feature is that you can compose these functions.
For example, you can easily get the \verb!v! variant of \verb!expl3! by
simply composing \cs{Name} and \cs{Value} functions:

\begin{demohigh}
\IntSet\lTmpaInt{123}
\Value{\Name{lTmpaInt}}
\end{demohigh}

\end{document}
