% -*- coding: utf-8 -*-
% !TEX program = lualatex
\documentclass[oneside]{book}

% -*- coding: utf-8 -*-
% !TEX program = lualatex

\usepackage[a4paper,margin=2.5cm]{geometry}

\newcommand*{\myversion}{2022F}
\newcommand*{\mydate}{Version \myversion\ (\the\year-\mylpad\month-\mylpad\day)}
\newcommand*{\mylpad}[1]{\ifnum#1<10 0\the#1\else\the#1\fi}

\setlength{\parindent}{0pt}
\setlength{\parskip}{4pt plus 1pt minus 1pt}

\usepackage{codehigh}

\colorlet{highback}{blue9}
%\CodeHigh{lite}
\CodeHigh{language=latex/latex2,style/main=highback,style/code=highback}
\NewCodeHighEnv{code}{style/main=gray9,style/code=gray9}
\NewCodeHighEnv{demo}{style/main=gray9,style/code=gray9,demo}

\usepackage{enumitem}

\NewDocumentCommand\MySubScript{m}{$_{#1}$}

\ExplSyntaxOn
\NewDocumentCommand\PrintVarList{m}{
  \clist_set:Nn \l_tmpa_clist {#1}
  \clist_map_inline:Nn \l_tmpa_clist
    {
      \token_to_str:N ##1 ~
    }
}
\NewDocumentCommand\RelaceChacters{m}{
  \tl_set:Nn \lTmpaTl {#1}
  \regex_replace_once:nnN { \_ } { \c{MySubScript} } \lTmpaTl
}
\NewDocumentCommand\RelaceUnderScoreAll{m}{
  \tl_set:Nn \lTmpaTl {#1}
  \regex_replace_all:nnN { \_ } { \c{_} } \lTmpaTl
  \lTmpaTl
}
\ExplSyntaxOff

\NewDocumentEnvironment{variable}{om}{
  \vspace{5pt}
  \begin{minipage}{\linewidth}
  \hrule\vspace{4pt}\obeylines%
  \begingroup
  \ttfamily\bfseries\color{azure3}
  \PrintVarList{#2}
  \endgroup
  \par\vspace{4pt}\hrule
  \end{minipage}\par\nopagebreak\vspace{4pt}
}{%
  \vspace{5pt}%
}

\NewDocumentEnvironment{function}{om}{
  \vspace{5pt}%
}{\vspace{5pt}}

\NewDocumentEnvironment{syntax}{}{%
  \begin{minipage}{\linewidth}
  \hrule\vspace{4pt}\obeylines%
}{%
  \par\vspace{4pt}\hrule
  \end{minipage}\par\nopagebreak\vspace{4pt}
}

\NewDocumentEnvironment{texnote}{}{}{}

\NewDocumentCommand\cs{O{}m}{%
  \texttt{\bfseries\color{purple3}\cBackslashStr\RelaceUnderScoreAll{#2}}%
}
\NewDocumentCommand\meta{m}{%
  \RelaceChacters{#1}%
  \textsl{$\langle$\ignorespaces\lTmpaTl\unskip$\rangle$}%
}
\NewDocumentCommand\Arg{m}{%
  \RelaceChacters{#1}%
  \texttt{\{}\textsl{$\langle$\ignorespaces\lTmpaTl\unskip$\rangle$}\texttt{\}}%
}

\NewDocumentCommand\pkg{m}{\textsf{#1}}

\NewDocumentCommand\nan{}{\texttt{NaN}}
\NewDocumentCommand\enquote{m}{``#1''}

\let\tn=\cs

\RenewDocumentCommand\emph{m}{%
  \underline{\textsl{#1}}%
}

\newenvironment{l3regex-syntax}
  {\begin{itemize}\def\\{\char`\\}\def\makelabel##1{\hss\llap{\ttfamily##1}}}
  {\end{itemize}}

\usepackage{shortvrb}

\NewDocumentCommand \MyMakeShortVerb {} {%
  \MakeShortVerb \|%
  \MakeShortVerb \"%
}
\NewDocumentCommand \MyDeleteShortVerb {} {%
  \DeleteShortVerb \|%
  \DeleteShortVerb \"%
}

\AtBeginDocument{\MyMakeShortVerb}
\AtEndDocument{\MyDeleteShortVerb}
\AddToHook{env/demohigh/before}{\MyDeleteShortVerb}
\AddToHook{env/demohigh/after}{\MyMakeShortVerb}

\usepackage{hologo}
\providecommand*\LuaTeX{\hologo{LuaTeX}}
\providecommand*\pdfTeX{\hologo{pdfTeX}}
\providecommand*\XeTeX{\hologo{XeTeX}}

\usepackage{amsmath}

\usepackage{hyperref}
\hypersetup{
  colorlinks=true,
  urlcolor=blue3,
  linkcolor=blue3,
}

\usepackage{functional}
%\Functional{scoping=false,tracing=true}

\CodeHigh{lite}

\begin{document}

\chapter{Text Processing (\texttt{Text})}

This module deals with manipulation of (formatted) text; such material is
comprised of a restricted set of token list content. The functions provided
here concern conversion of textual content for example in case changing,
%generation of bookmarks and extraction to tags.
Begin-group and end-group tokens in the \meta{text}
are normalized and become \verb|{| and \verb|}|, respectively.

\section{Case Changing}

These case changing functions are designed to work with Unicode input only.
As such, UTF-8 input is assumed for \emph{all} engines.
When used with XeTeX or LuaTeX a full range of Unicode transformations
are enabled. Specifically, the standard mappings
here follow those defined by the \href{http://www.unicode.org}
{Unicode Consortium} in \texttt{UnicodeData.txt} and
\texttt{SpecialCasing.txt}. In the case of $8$-bit engines, mappings
are provided for characters which can be represented in output typeset
using the \verb|T1|, \verb|T2| and \verb|LGR| font encodings.
Thus for example \texttt{ä} can be
case-changed using pdfTeX.  For pTeX only the ASCII range is
covered as the engine treats input outside of this range as east Asian.

%Importantly, notice that these functions are intended for working with
%user \emph{text for typesetting}. For case changing programmatic data see
%the \pkg{Str} module and discussion there of \cs{StrLowercase},
%\cs{StrUppercase} and \cs{StrFoldcase}.

\begin{function}{\TextExpand}
\begin{syntax}
\cs{TextExpand} \Arg{text}
\end{syntax}
Takes user input \meta{text} and expands the content.
Protected commands (typically formatting) are left in place,
and no processing takes place of math mode material.
%(as delimited by pairs given in \cs{l_text_math_delims_tl}
%or as the argument to commands listed in \cs{l_text_math_arg_tl}).
Commands which are neither engine- nor \LaTeX{} protected are expanded exhaustively.
%Any commands listed in \cs{l_text_expand_exclude_tl},
%\cs{l_text_accents_tl} and \cs{l_text_letterlike_tl} are excluded from expansion.
\end{function}

\begin{function}{\TextLowercase,\TextUppercase,\TextTitlecase,\TextTitlecaseFirst}
\begin{syntax}
\cs{TextLowercase}  \Arg{tokens}
\cs{TextUppercase}  \Arg{tokens}
\cs{TextTitlecase}  \Arg{tokens}
\cs{TextTitlecaseFirst}  \Arg{tokens}
\end{syntax}
Takes user input \meta{text} first applies \cs{TextExpand}, then
transforms the case of character tokens as specified by the
function name. The category code of letters are not changed by this
process (at least where they can be represented by the engine as a single
token: $8$-bit engines may require active characters).
\par
Upper- and lowercase have the obvious meanings. Titlecasing may be regarded
informally as converting the first character of the \meta{tokens} to
uppercase and the rest to lowercase. However, the process is more complex
than this as there are some situations where a single lowercase character
maps to a special form, for example \texttt{ij} in Dutch which becomes
\texttt{IJ}.
\par
For titlecasing, note that there are two functions available. The
function \cs{TextTitlecase} applies (broadly) uppercasing to the first
letter of the input, then lowercasing to the remainder. In contrast,
\cs{TextTitlecaseFirst} \emph{only} carries out the uppercasing operation,
and leaves the balance of the input unchanged.
%Determining whether non-letter characters at the start of text should switch
%from upper- to lowercasing is controllable.
%When \cs{l_text_titlecase_check_letter_bool} is \texttt{true},
%characters which are not letters (category code $11$) are
%left unchanged and \enquote{skipped}: the first \emph{letter} is uppercased.
%(With $8$-bit engines, this is extended to active characters which form
%part of a multi-byte letter codepoint.) When
%\cs{l_text_titlecase_check_letter_bool} is \texttt{false}, the first
%character is uppercased, and the rest lowercased, irrespective of the nature
%of the character.
\par
Case changing does not take place within math mode material. For example:
\begin{demohigh}
\TextUppercase {Text $y=mx+c$ with {Braces}}
\end{demohigh}
\begin{demohigh}
\TextLowercase {Text $Y=mX+c$ with {Braces}}
\end{demohigh}
%The arguments of commands listed in \cs{l_text_case_exclude_arg_tl}
%are excluded from case changing; the latter are entirely non-textual
%content (such as labels).
\end{function}

\begin{function}{\TextLangLowercase,\TextLangUppercase,\TextLangTitlecase,\TextLangTitlecaseFirst}
\begin{syntax}
\cs{TextLangLowercase} \Arg{language} \Arg{tokens}
\cs{TextLangUppercase} \Arg{language} \Arg{tokens}
\cs{TextLangTitlecase} \Arg{language} \Arg{tokens}
\cs{TextLangTitlecaseFirst} \Arg{language} \Arg{tokens}
\end{syntax}
Takes user input \meta{text} first applies \cs{TextExpand}, then
transforms the case of character tokens as specified by the
function name. The category code of letters are not changed by this
process (at least where they can be represented by the engine as a single
token: $8$-bit engines may require active characters).
\par
These conversions are language-sensitive, and follow Unicode Consortium guidelines.
Currently, the languages recognised for special handling are as follows.
\begin{itemize}
\item Azeri and Turkish (\texttt{az} and \texttt{tr}).
  The case pairs I/i-dotless and I-dot/i are activated for these
  languages. The combining dot mark is removed when lowercasing
  I-dot and introduced when upper casing i-dotless.
\item German (\texttt{de-alt}).
  An alternative mapping for German in which the lowercase
  \emph{Eszett} maps to a \emph{gro\ss{}es Eszett}. Since there is
  a \verb|T1| slot for the \emph{gro\ss{}es Eszett} in \verb|T1|, this
  tailoring \emph{is} available with pdfTeX as well as in the
  Unicode \TeX{} engines.
\item Greek (\texttt{el}).
  Removes accents from Greek letters when uppercasing; titlecasing
  leaves accents in place. (At present this is implemented only
  for Unicode engines.)
\item Lithuanian (\texttt{lt}).
  The lowercase letters i and j should retain a dot above when the
  accents grave, acute or tilde are present. This is implemented for
  lowercasing of the relevant uppercase letters both when input as
  single Unicode codepoints and when using combining accents. The
  combining dot is removed when uppercasing in these cases. Note that
  \emph{only} the accents used in Lithuanian are covered: the behaviour
  of other accents are not modified.
\item Dutch (\texttt{nl}).
  Capitalisation of \texttt{ij} at the beginning of titlecased
  input produces \texttt{IJ} rather than \texttt{Ij}. The output
  retains two separate letters, thus this transformation \emph{is}
  available using pdfTeX.
\end{itemize}
\end{function}

\end{document}
